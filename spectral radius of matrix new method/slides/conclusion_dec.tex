\documentclass{beamer}
%\usetheme{Warsaw}
\usepackage{times}  % fonts are up to you
\usepackage{amssymb, amsmath, mathrsfs}
\usepackage{multicol}
\usepackage{bm}
\usepackage{ulem}


\usepackage{fontspec}
\usepackage{xeCJK}
\setCJKmainfont{微軟正黑體}
\XeTeXlinebreaklocale "zh"
\XeTeXlinebreakskip = 0pt plus 1pt

% THEOREMS ---------------------------------------------------------------
\theoremstyle{plain}
% block中的字會變成數學體, 斜體字
\newtheorem{thm}{Theorem}[section]
\newtheorem{cor}[thm]{Corollary}
\newtheorem{lem}[thm]{Lemma}
\newtheorem{prop}[thm]{Proposition}
\newtheorem{remark}[thm]{Remark}

\newtheorem{eg}[thm]{Example}
\newtheorem{conj}[thm]{Conjecture}

\theoremstyle{definition}
% block中的字是正常字體
\newtheorem{ex}[thm]{Exercise}
\newtheorem{defn}[thm]{Definition}
\newtheorem{prob}[thm]{Problem}
\newtheorem{exam}[thm]{Example}
\newtheorem{rem}[thm]{Remark}

\newtheorem{algo}[thm]{Algorithm}
%--------------------------------------------------------------------------
\setbeamertemplate{footline}[page number]{}
% 加頁碼, 到資料夾裡可以改得更徹底
\setbeamertemplate{navigation symbols}
% 加這行會把outline顯示在最上層

\begin{document}


\title[]{TBA}
\author[]{Ke-Han Chen\\} % 作者
\institute[Dep. of A. Math., NCTU]{Department of Applied Mathematics\\National Chiao Tung University}
\date{October 29th, 2018}
%\date{\today} % 日期

\begin{frame}
\maketitle
\end{frame}

\begin{frame}{\bf Preliminaries}
    \begin{thm}
        If $C$ is nonnegative square matrix, then the following (i)-(iii) hold.
    \begin{enumerate}
        \item[(i)] The spectral radius $\rho(C)$ is an eigenvalue of $C$ with a corresponding nonnogative right eigenvector and a corresponding nonnegative left eigenvector.
        \item[(ii)] If there exists a column vector $v$ > 0, and a nongative numver $\lambda$ such that $Cv \leq \lambda v$, then $\rho(C) \leq \lambda$.
        \item[(iii)] If there exists a column vector $v \geq 0, v \neq 0$ and a nonnegative number $\lambda$ such that $Cv \geq \lambda v$, then $\rho(C) \leq \lambda$.
    \end{enumerate}

    \end{thm}
\end{frame}

\begin{frame}{\bf Preliminaries}

The following theorem is from [].

\begin{thm}
 Let $C=(c_{ij})$, $C'=(c'_{ij})$, $P$ and $Q$ be  $n\times n$ matrices.
Assume that
\begin{enumerate}
\item[(i)]    $PCQ\leq PC'Q$;
\item[(ii)]  there exist a nonnegative column vector $u=(u_1, u_2, \ldots, u_n)^T$  and a scalar $\lambda'\in \mathbb{R}$ such that $\lambda'$ is an eigenvalue of $C'$ with associated eigenvector $Qu$;
\item[(iii)] there exist a nonnegative row vector $v^T=(v_1, v_2, \ldots, v_n)$  and a scalar $\lambda\in \mathbb{R}$
such that $\lambda$ is an eigenvalue of $C$ with associated  left eigenvector $v^TP$; and
\item[(iv)] $v^TPQu>0.$
\end{enumerate}
 Then $\lambda\leq \lambda'$.
    Moreover, $\lambda=\lambda'$ 
if and only if
\begin{equation*}
(PC'Q)_{ij}=(PCQ)_{ij}\qquad \hbox{for~}1\leq i, j\leq n \hbox{~with~} v_i\ne 0 \hbox{~and~} u_j\ne 0.
\end{equation*}
\end{thm}
\end{frame}

\begin{frame}{\bf Proof}
            Multiplying the nonnegative vector $u$ in (ii) to the right of both terms of  (i),
        \begin{equation}\label{e1}
            PCQu\leq PC'Qu=\lambda'PQu.
        \end{equation}
        Multiplying the nonnegative left eigenvector $v^T$ of $C$ for $\lambda$ in assumption (iii) to the left of all terms  in (\ref{e1}), we have
        \begin{equation}\label{e2}
            \lambda v^TPQu=v^TPCQu\leq v^TPC'Qu=\lambda' v^TPQu.
        \end{equation}
        Now delete the positive term $v^TPQu$ by assumption (iv) to obtain $\lambda\leq \lambda'$ and finish the proof of the first part.
\end{frame}

\begin{frame}{\bf Proof (Continue)}
        Assume that $\lambda=\lambda'$, so the inequality in (\ref{e2}) is an equality.  Especially $(PCQu)_i=(PC'Qu)_i$ for any $i$ with $v_i\not=0.$ Hence $(PCQ)_{ij}=(PC'Q)_{ij}$ for any $i$ with $v_i\not=0$ and any $j$ with $u_j\not=0.$
        Conversely, (\ref{e2}) implies $$v^TPCQu=\sum_{i,j} v_i(PCQ)_{ij}u_j=\sum_{i,j} v_i(PC'Q)_{ij}u_j=v^TPC'Qu,$$ so
            $\lambda=\lambda'$ by (\ref{e2}).
\end{frame}

\begin{frame}{\bf Our Method}


 Throughout fix $k\in [n-1]$. Let $E_{kn}$ denote the $n\times n$ binary matrix with a unique $1$ appearing in the  position $k,n$ of $E_{kn}$. We will apply the previous theorem with $P=I$ and $$Q=I+E_{kn}=\begin{pmatrix}
1 &  & & &  & 0 \\
 & 1 &  &      &  &  \\
 &  & \ddots & &  & 1 \\
 &  &        & &  &  \\
  &  & & & 1 &  \\
0 &  & & &  & 1 \\
\end{pmatrix}.$$

\end{frame}

\begin{frame}{\bf $k$-rooted vector}

\begin{defn}
 A column vector $v'=(v'_1,v'_2,\ldots,v'_n)^T$ is called {\it $k$-rooted}  if $v'_{j} \geq 0$ for $1 \leq  j \leq n$ and $v'_k\geq v'_n.$
\end{defn}
\bigskip

The following Lemma is immediate from the above definition.
\bigskip

\begin{lem}{vector rooted lemma}
If $u=(u_1, u_2, \ldots, u_n)^T$ and $v'=(v'_1, v'_2, \ldots, v'_n):=Qu=(u_1,\ldots, u_{k-1},u_k+u_n, u_{k+1}, \ldots,  u_n)^T$, then
\begin{enumerate}
\item[(i)] $v'$ is $k$-rooted  if and only if  $u$ is nonnegative;
\item[(ii)] $u_k>0$ if and only if $v'_k>v'_n$.
\end{enumerate}
\qed
\end{lem}

\end{frame}

\begin{frame}{\bf Our first result}

\begin{thm}
    Let $C=(c_{ij})$, $C'=(c'_{ij})$ be  $n\times n$ matrices.
Assume that
\begin{enumerate}
\item[(i)]   $C[-|n)\leq C'[-|n)$ and $c_{ik}+c_{in}\leq c'_{ik}+c'_{in}$ for all $1\leq i\leq n$;
\item[(ii)] there exist a $k$-rooted vector $v'=(v'_1, v'_2, \ldots, v'_n)^T$ and a scalar $\lambda'\in \mathbb{R}$
such that $\lambda'$ is an eigenvalue of $C'$ with associated eigenvector $v'$;
\item[(iii)] there exists a nonnegative vector $v^T=(v_1, v_2, \ldots, v_n)$ and a scalar $\lambda\in \mathbb{R}$ such that $\lambda$ is an eigenvalue of $C$ with associated left eigenvector $v^T$;
\item[(iv)] $v^Tv'>0.$
\end{enumerate}
 Then $\lambda\leq \lambda'$.
Moreover, $\lambda=\lambda'$
if and only if
\begin{enumerate}
\item[(a)] $c_{ik}+c_{in}=c'_{ik}+c'_{in}$\qquad for $1\leq i\leq n$ with $v_i\not=0$ and $v'_n\not=0;$
\item[(b)]
$c'_{ij}=c_{ij}\qquad \hbox{for~}1\leq i\leq n,~1\leq j\leq n-1 \hbox{~with~} v_i\ne 0 \hbox{~and~} v'_j> v'_n.$
\end{enumerate} \qed
\end{thm}
\end{frame}



\begin{frame}{\bf $k$-rooted matrix}

\begin{defn}[(k,n)-sum]
For a matrix $C'=(c'_{ij})$ of $n$ columns, the $(k, n)$-{\it sum} vector of $C'$ is the vector of the sum of the $k$-th and  $n$-th columns of $C'$, where $k\leq n-1$.
\end{defn}

\begin{defn}[k-rooted matrix]
A  matrix $C'=(c'_{ij})$ is called {\it $k$-rooted}  if its  columns and its $(k, n)$-sum vector are all $k$-rooted except the last column of $C'$.
\end{defn}

\end{frame}

\begin{frame}

\begin{thm}
If $C'$ is a $k$-rooted matrix, then $Q^{-1}C'Q$ is nonnegative, $\rho(C')$ is an eigenvalue of $C'$,
and $C'$ has a $k$-rooted eigenvector $v'=Qu$ for $\rho(C')$,
where $u$ is a nonnegative eigenvector of $Q^{-1}C'Q$ for $\rho(C')$.
    Moreover, with $v'=(v'_1, v'_2, \ldots, v'_n)^T$, $r'_{i} = C'_{ik} + C'_{in}$
    the following (i)-(ii) hold.
\begin{enumerate}
    \item[(i)] If $C' (k|n)$ is positive, then $v'$ is positive.
    \item[(ii)] If  $C' (k|n)$ is positive and  $r'_i> r'_n$ for all $1\leq i\leq n-1$, then $v'_j>v'_n$ for all $1\leq j\leq n-1.$
\end{enumerate}

\end{thm}

Let $\mathbb{R}$ and $\mathbb{C}$ denote the field of real numbers and complex numbers respectively.


\begin{defn}
    Let $C$ be an $n \times n$ real matrix, and $u \in \mathbb{R}^n$ be a nonzero column vector. The scalar $\lambda \in \mathbb{C}$ is an $\it{eigenvalue}$ of C corresponding to the $\it{eigenvector}$ $u,$  if $Cu = \lambda u.$
   
\end{defn}

\begin{defn}

When $C$ is an $n \times n$ real matrix, the $\textit {spectral radius} $ $\rho(C)$ of $C$ is defined by
$$\rho(C):=\max\{~|\lambda|~\ |~~\lambda\hbox{ is an eigenvalue of $C$}\},$$
where $|\lambda|$ is the magnitude of complex number $\lambda.$
\end{defn}

\end{frame}

\begin{frame}

\begin{proof}
(i) suppose that $C' [n|n)$ is positive and $v'_n=0$. Then
$$\sum_{j=1}^{n-1}c'_{nj}v'_j=\sum_{j=1}^nc'_{nj}v'_j=(C'v')_n=\rho(C')v'_n=0.$$
Hence $v'$ is a zero vector since $c'_{nj}>0$ for $j\leq n-1$, a contradiction. So $v'_n>0$ and $v'>0$ since $v'$ is rooted.


(ii) The assumptions imply that the matrix $Q^{-1}C'Q$ in (\ref{e3.1}) is irreducible.
Hence $u$ is positive. By Lemma~\ref{l3.15}(ii), $v'_j>v'_n$ for $1\leq j<n.$
\end{proof}



\end{frame}

\begin{frame}{background}

    \begin{enumerate}
        \item[(i)] $C^{ '}$*Qu = $\lambda * Qu $ = Q * $\lambda u$  ....times $Q^{-1}$ to the left
        \item[(ii)] $Q^{-1} C^{ '}Q$* u = $\lambda * u$
    \end{enumerate}

    Here the $\lambda$ is expected to be the spectral radius of $C^{ '}$ so we wish that $Q^{-1}C^{ '}Q$ > 0, such that this condition $Q^{-1}C^{ '}Q$ > 0 allows that \\
    \begin{enumerate}
        \item[(i)] $ C^{ '}$ is k-rooted $\Leftrightarrow Q^{-1}C^{ '}Q > 0$
        \item[(ii)] $ C^{ '}$ is k-rooted $\Rightarrow C^{ '}$ has a rooted eigenvector
    \end{enumerate}
	\begin{proof}
        if $C^{ '}$ is k-rooted, all columns and (k,n)-sum are rooted except the last one and then $c'_{kj}-c'_{nj} \geq 0$ and $c'_{ij} \geq 0$ for $i \neq k, i < n$, for $1 \leq j \leq n-1$. So all columns except the last of $Q^{-1}C^{ '}Q$ is nonnegative.
    \end{proof}


\end{frame}

\begin{frame}{\bf Remark}

$$Q^{-1}=I-E_{kn}=\begin{pmatrix}
1 &  & & &  & 0 \\
 & 1 &  &      &  &  \\
 &  & \ddots & &  & -1 \\
 &  &        & &  &  \\
  &  & & & 1 &  \\
0 &  & & &  & 1 \\
\end{pmatrix}.$$

The matrix $Q^{-1}C'Q$ is
$$\begin{pmatrix}
c'_{11}     & c'_{12} & \cdots     & c'_{1\ n-1} & c'_{1k}+c'_{1n} \\
\vdots \\
c'_{k-11}     & c'_{k-1 2}           & \cdots     & c'_{k-1 n-1} & c'_{k-1k}+c'_{k-1n} \\
c'_{k1}-c'_{n1} & c'_{k2}-c'_{n2} &\cdots      &c'_{kn-1}-c'_{nk-1}& c'_{kk}+c'_{kn}-c'_{nk}-x'_{nn}\\
c'_{k+11}     & c'_{k+12}           & \cdots     & c'_{k+1\ n-1} & c'_{k+1k}+c'_{k+1n} \\
\vdots              & \vdots & \ddots              & \vdots & \vdots \\
c'_{n1}             & c'_{n2} & \cdots             & c'_{n\ n-1} & c'_{nk}+c'_{nn} \\
\end{pmatrix}.
$$
\end{frame}

\begin{frame}
    \begin{proof}
        $C^{ '}$ is k-rooted also implies that (k,n)-sum vector is k-rooted,
	the last column of $Q^{-1}C^{ '}Q$ is nonnegative.
	conversely, if all the columns of $Q^{-1}C^{ '}Q$ are nonegative, $C^{ '}$ is k-rooted.
    \end{proof}
	second, $C^{ '}$ is k-rooted implies that the two postulations of the lemma, $C' (k|n)$ is positive and  $r'_i> r'_n$ for all $1\leq i\leq n-1$ also note that the conclusion of the lemma !!adapt it!! is $v'$ is positive. and $v'_j>v'_n$ for all $1\leq j\leq n-1.$ ,that is, the eigenvector of $C^{ '}$ is  k-rooted.

\end{frame}

\begin{frame}{application and some summantion}
    \begin{thm}
        maybe the form of Q, is able to generalize the condition of some matrices with same eigenvalues\\
        or the choice of $C^{ '}$ and Q might affect the spectral radius.
        the form of eigenvector transformed but the relation in reader's mind remain, so he or she doesn't comprehense the condition difference immediately. \\
	$Q^{-1}C^{ '}Q > 0 \Rightarrow C^{ '}$ has a rooted eigenvector

    \end{thm}
\end{frame}


\begin{frame}

\begin{thm}
Let $C$ be an $n\times n$ nonnegative matrix. For $1\leq i \leq n$ and $1\leq j\leq n-1$, choose $r'_i$, $c'_{ij}$
such that $c'_{ij}\geq c_{ij}$, $r'_i\geq c_{ik}+c_{in}$ and $c'_{kj}\geq c'_{nj}>0$, and 
let $c'_{in}:=r'_i-c'_{ik}$. Then the $n\times n$ matrix $C'=(c'_{ij})$ has a positive $k$-rooted eigenvector for $\rho(C')$ and $\rho(C)\leq \rho(C')$. 
\end{thm}

\end{frame}

\begin{frame}{\bf Proof}



\end{frame}


\begin{frame}{\bf Example}
For the following $4\times 4$ matrix  
$$C=\begin{pmatrix}
0 & 0 & 1 & 1\\ 
1 & 0 & 0 & 1\\ 
1 & 1 & 0 & 0\\ 
1 & 1 & 1 & 0
\end{pmatrix},$$ 
we choose  
$$C'=\begin{pmatrix}
0 & 0 & 1 & 1\\
1 & 0 & 1 &  0\\
1 & 1 & 0 & 0\\
1 & 1 & 1 & 0
\end{pmatrix}.$$ 
Then 
$\rho(C)\leq \rho(C')$ by previous theorem. 
\end{frame}



\begin{frame}{conclusion}
    \begin{thm}
    \end{thm}
\end{frame}
\end{document}
