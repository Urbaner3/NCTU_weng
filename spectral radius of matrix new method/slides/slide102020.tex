\documentclass{beamer}
%\usetheme{Warsaw}
\usepackage{times}  % fonts are up to you
\usepackage{amssymb, amsmath,amsthm, mathrsfs}
\usepackage{multicol}
\usepackage{bm}
\usepackage{ulem}


\usepackage{fontspec}
\usepackage{xeCJK}
\setCJKmainfont{微軟正黑體}
\XeTeXlinebreaklocale "zh"
\XeTeXlinebreakskip = 0pt plus 1pt

% THEOREMS ---------------------------------------------------------------
\theoremstyle{plain}
% block中的字會變成數學體, 斜體字
\newtheorem{thm}{Theorem}[section]
\newtheorem{cor}[thm]{Corollary}
\newtheorem{lem}[thm]{Lemma}
\newtheorem{prop}[thm]{Proposition}
\newtheorem{remark}[thm]{Remark}

\newtheorem{eg}[thm]{Example}
\newtheorem{conj}[thm]{Conjecture}

\theoremstyle{definition}
% block中的字是正常字體
\newtheorem{ex}[thm]{Exercise}
\newtheorem{defn}[thm]{Definition}
\newtheorem{prob}[thm]{Problem}
\newtheorem{exam}[thm]{Example}
\newtheorem{rem}[thm]{Remark}

\newtheorem{algo}[thm]{Algorithm}
%--------------------------------------------------------------------------
\setbeamertemplate{footline}[page number]{}
% 加頁碼, 到資料夾裡可以改得更徹底
\setbeamertemplate{navigation symbols}
% 加這行會把outline顯示在最上層

\usepackage{comment}

\begin{document}


\title[]{TBA}
\author[]{Ke-Han Chen} % 作者
\institute[Dep. of A. Math., NCTU]{Department of Applied Mathematics\\National Chiao Tung University}
% \date{October 6th, 2020}% 日期
\date{\today} % 日期

\begin{frame}
    \maketitle
\end{frame}

\begin{frame}{Preliminaries}
    

    Let $\mathbb{R}$ and $\mathbb{C}$ denote the field of real numbers and complex numbers respectively.
    \begin{defn}
        Let $C$ be an $n \times n$ real nonnegative matrix, and $u \in \mathbb{R}^n$ be a nonzero column vector. The scalar $\lambda \in \mathbb{C}$ is an $\it{eigenvalue}$ of C corresponding to the $\it{eigenvector}$ $u,$  if $Cu = \lambda u.$
        %Also, $C$ has an left eigenvector $v^T$, when $v^TC=\lambda v^T$.
    \end{defn}
    \begin{defn}
        When $C$ is an $n \times n$ real matrix, the $\textit {spectral radius} $ $\rho(C)$ of $C$ is defined by
        $$\rho(C):=\max\{~|\lambda|~\ |~~\lambda\hbox{ is an eigenvalue of $C$}\},$$
        where $|\lambda|$ is the magnitude of complex number $\lambda.$
    \end{defn}
\end{frame}

\begin{frame}{\bf Preliminaries}
    \begin{thm} 
        %\bf{Perron–Frobenius theorem} \\
        If $C$ is nonnegative square matrix, then the spectral radius $\rho(C)$ is an eigenvalue of $C$ with a corresponding nonnogative right eigenvector and a corresponding nonnegative left eigenvector.
    \end{thm}
\end{frame}


\begin{frame}

    We are interested in spectral radius of the following matrix associated with a simple graph.

    \begin{defn}
    Given an undirected graph G, the$\textit{ adjacency matrix}$ of G is the square matrix $A = (a_{ij})$ indexed by vertices of G, and
    \[a_{ij} =\begin{cases}
    1, \text{if $i$ is adjacent to $j$}, \\
    0, \text{otherwise.}
    \end{cases}
    \]
    \end{defn}
    
    \begin{defn}
    Given an undirected graph G, the $\textit{spectral radius}$  $\rho(G) $ of G is the spectral radius of the adjacency matrix of G.
    \end{defn}

\end{frame}

\begin{frame}
    We introduce a notation of submatrix, which is taken from some columns and some rows of a matrix.
    \begin{defn}
        For a matrix $C=(c_{ij})$ and subsets $\alpha$, $\beta$ of row indices and column indices of 
        $C$ respectively,  We use $C[\alpha|\beta]$ to denote the submatrix of $C$ with size
         $ |\alpha| \times |\beta| $ that has entries $c_{ij}$ for $i\in \alpha$ and $j\in\beta$,
    \end{defn}

\end{frame}

\begin{frame}
    

%The following theorem is from [].\cite{content}

    \begin{thm}\label{pre_thm}
        Let $C=(c_{ij})$, $C'=(c'_{ij})$, $P$ and $Q$ be  $n\times n$ matrices.
        Assume that
        \begin{enumerate}
            \item[(i)]    $PCQ\leq PC'Q$;\label{Q1}
            \item[(ii)]  there exist a nonnegative column vector $u=(u_1, u_2, \ldots, u_n)^T$  and a scalar $\lambda'\in \mathbb{R}$ such that $\lambda'$ is an eigenvalue of $C'$ with associated eigenvector $Qu$;
            \item[(iii)] there exist a nonnegative row vector $v^T=(v_1, v_2, \ldots, v_n)$  and a scalar $\lambda\in \mathbb{R}$
            such that $\lambda$ is an eigenvalue of $C$ with associated  left eigenvector $v^TP$; and
            \item[(iv)] $v^TPQu>0.$
            \end{enumerate}
            Then $\lambda\leq \lambda'$.
                Moreover, $\lambda=\lambda'$ 
            if and only if
                \begin{equation}
                    \label{e3}
            (PC'Q)_{ij}=(PCQ)_{ij}\qquad \hbox{for~}1\leq i, j\leq n \hbox{~with~} v_i\ne 0 \hbox{~and~} u_j\ne 0.
        \end{equation}
    \end{thm}%
\end{frame}

\begin{frame}
    % \frametitle{ Proof}
            Multiplying the nonnegative vector $u$ in (ii) to the right of both terms of  (i),
        \begin{equation}\label{e1}
            PCQu\leq PC'Qu=\lambda'PQu.
        \end{equation}
        Multiplying the nonnegative left eigenvector $v^T$ of $C$ for $\lambda$ in assumption (iii) to the left of all terms  in (\ref{e1}), we have
        \begin{equation}\label{e2}
            \lambda v^TPQu=v^TPCQu\leq v^TPC'Qu=\lambda' v^TPQu.
        \end{equation}
        Now delete the positive term $v^TPQu$ by assumption (iv) to obtain $\lambda\leq \lambda'$ and finish the proof of the first part.
\end{frame}

\begin{frame}
    \frametitle{ Proof (Continue)}
        Assume that $\lambda=\lambda'$, so the inequality in (\ref{e2}) is an equality.  Especially $(PCQu)_i=(PC'Qu)_i$ for any $i$ with $v_i\not=0.$ Hence, $(PCQ)_{ij}=(PC'Q)_{ij}$ for any $i$ with $v_i\not=0$ and any $j$ with $u_j\not=0.$ 
         Conversely, (\ref{e3}) implies $$v^TPCQu=\sum_{i,j} v_i(PCQ)_{ij}u_j=\sum_{i,j} v_i(PC'Q)_{ij}u_j=v^TPC'Qu,$$ so
            $\lambda=\lambda'$ by (\ref{e2}).
\end{frame}

\begin{frame}
    \frametitle{ Our Method}


 Throughout fix $k\in [n-1]$. Let $E_{kn}$ denote the $n\times n$ binary matrix with a unique $1$ appearing in the  position $k,n$ of $E_{kn}$. We will apply the previous theorem with $P=I$ and $$Q=I+E_{kn}=\begin{pmatrix}
1 &  & & &  & 0 \\
 & 1 &  &      &  &  \\
 &  & \ddots & &  & 1 \\
 &  &        & &  &  \\
  &  & & & 1 &  \\
0 &  & & &  & 1 \\
\end{pmatrix}.$$

\end{frame}

% \begin{frame}
    \frametitle{ $k$-rooted vector}

\begin{defn}
 A column vector $v'=(v'_1,v'_2,\ldots,v'_n)^T$ is called {\it $k$-rooted}  if $v'_{j} \geq 0$ for $1 \leq  j \leq n$ and $v'_k\geq v'_n.$
\end{defn}
\bigskip

The following Lemma is immediate from the above definition.
\bigskip

\begin{lem}
\begin{equation}
    \textit{vector rooted lemma}\label{v_rooted}
\end{equation}
If $u=(u_1, u_2, \ldots, u_n)^T$ and $v'=(v'_1, v'_2, \ldots, v'_n):=Qu=(u_1,\ldots, u_{k-1},u_k+u_n, u_{k+1}, \ldots,  u_n)^T$, then
\begin{enumerate}
\item[(i)] $v'$ is $k$-rooted  if and only if  $u$ is nonnegative;
\item[(ii)] $u_k>0$ if and only if $v'_k>v'_n$.
\end{enumerate}
\qed
\end{lem}

% \end{frame}

% \begin{frame}
    \frametitle{ Our first result}

\begin{thm}\label{thm_main}
    Let $C=(c_{ij})$, $C'=(c'_{ij})$ be  $n\times n$ matrices.
Assume that
\begin{enumerate}
\item[(i)]   $C[[n]|[n-1]]\leq C'[[n]|[n-1]]$ and $c_{ik}+c_{in}\leq c'_{ik}+c'_{in}$ for all $1\leq i\leq n$;
\item[(ii)] there exist a $k$-rooted vector $v'=(v'_1, v'_2, \ldots, v'_n)^T$ and a scalar $\lambda'\in \mathbb{R}$
such that $\lambda'$ is an eigenvalue of $C'$ with associated eigenvector $v'$;
\item[(iii)] there exists a nonnegative vector $v^T=(v_1, v_2, \ldots, v_n)$ and a scalar $\lambda\in \mathbb{R}$ such that $\lambda$ is an eigenvalue of $C$ with associated left eigenvector $v^T$;
\item[(iv)] $v^Tv'>0.$
\end{enumerate}
 Then $\lambda\leq \lambda'$.
Moreover, $\lambda=\lambda'$
if and only if
\begin{enumerate}
\item[(a)] $c_{ik}+c_{in}=c'_{ik}+c'_{in}$\qquad for $1\leq i\leq n$ with $v_i\not=0$ and $v'_n\not=0;$
\item[(b)]
$c'_{ij}=c_{ij}\qquad \hbox{for~}1\leq i\leq n,~1\leq j\leq n-1 \hbox{~with~} v_i\ne 0 \hbox{~and~} v'_j> v'_n.$
\end{enumerate} \qed
\end{thm}
% \end{frame}



% \begin{frame}{}
   
    \begin{proof}
        The proof is based on Theorem (\ref{pre_thm}) with $P = I$ and $Q = I + E_{kn} \text{ in } (\ref{Q1})  $.%!!!!!!!! 
        The assumption (i) $PCQ\leq PC'Q$ of Theorem (\ref{pre_thm}) holds by the condition (i) of this theorem. 
        Let $u = Q^{-1}v'$. Then u is nonnegative and $C'Qu = \lambda' Qu$ by the condition (ii) and 
        Lemma 2.2(i). Hence the assumption (ii) of Theorem (\ref{pre_thm}) holds. The assumptions (iii) and (iv) 
        of Theorem (\ref{pre_thm}) clearly hold by conditions (iii),(iv) of this theorem since $P = I$ and $v'= Qu$  
        Hence $\lambda \leq \lambda' $ by the conclusion of Theorem (\ref{pre_thm}). Moreover $\lambda = \lambda'$ 
        if and only if (\ref{e3}) holds, and this is equivalent to conditions (a),(b) of this theorem. 
        \end{proof}   
    

% \end{frame}

% \begin{frame}
    {\bf $k$-rooted matrix}
    \begin{defn}[(k,n)-sum]
        For a matrix $C'=(c'_{ij})$ of $n$ columns, the $(k, n)$-{\it sum} vector of $C'$ is the vector of the sum of the $k$-th and  $n$-th columns of $C'$, where $k\leq n-1$.
    \end{defn}

    \begin{defn}[k-rooted matrix]\label{m_rooted}
        A  matrix $C'=(c'_{ij})$ is called {\it $k$-rooted}  if its  columns and its $(k, n)$-sum vector are all $k$-rooted except the last column of $C'$.
    \end{defn}
% \end{frame}

% \begin{frame}

    \begin{thm} \label{lma_m_rooted}
        Let $C'=(c'_{ij})$ be an $n\times n$ nonnegative matrix. Then the following (i)-(iii) hold.
            \begin{enumerate}
                \item[(i)]$C'$ is a $k$-rooted matrix, if and only if, $Q^{-1}C'Q$ is nonnegative.
                \item[(ii)]Assume that $C'$ is $k$-rooted and let $u$ be a nonnegative eigenvector of $Q^{-1}C'Q$ for $\rho(C')$. Then  $C'$ has a $k$-rooted eigenvector $v'=Qu$ for $\rho(C')$. 
                \item[(iii)] $\rho(C')$ = $\rho(Q^{-1}C'Q)$
            \end{enumerate}
        \end{thm}

% \end{frame}

\begin{frame}
    \frametitle{Remark}

    $$Q^{-1}=I-E_{kn}=\begin{pmatrix}
    1 &  & & &  & 0 \\
    & 1 &  &      &  &  \\
    &  & \ddots & &  & -1 \\
    &  &        & &  &  \\
    &  & & & 1 &  \\
    0 &  & & &  & 1 \\
    \end{pmatrix}.$$

    The matrix $Q^{-1}C'Q$ is
    $$\begin{pmatrix}
    c'_{11}     & c'_{12} & \cdots     & c'_{1\ n-1} & c'_{1k}+c'_{1n} \\
    \vdots \\
    c'_{k-11}     & c'_{k-1 2}           & \cdots     & c'_{k-1 n-1} & c'_{k-1k}+c'_{k-1n} \\
    c'_{k1}-c'_{n1} & c'_{k2}-c'_{n2} &\cdots      &c'_{kn-1}-c'_{nk-1}& c'_{kk}+c'_{kn}-c'_{nk}-x'_{nn}\\
    c'_{k+11}     & c'_{k+12}           & \cdots     & c'_{k+1\ n-1} & c'_{k+1k}+c'_{k+1n} \\
    \vdots              & \vdots & \ddots              & \vdots & \vdots \\
    c'_{n1}             & c'_{n2} & \cdots             & c'_{n\ n-1} & c'_{nk}+c'_{nn} \\
    \end{pmatrix}.
    $$
\end{frame}

\begin{frame}{Remark}

	\begin{proof}
        (i) is immediate from Definition~\ref{m_rooted} and the observation that   
        $$Q^{-1}=I-E_{kn}=\begin{pmatrix}
        1 &  & & &  & 0 \\
         & 1 &  &      &  &  \\
         &  & \ddots & &  & -1 \\
         &  &        & &  &  \\
          &  & & & 1 &  \\
        0 &  & & &  & 1 \\
        \end{pmatrix},$$
    \end{proof}
    
\end{frame}

\begin{frame}
    \begin{proof}
        and $Q^{-1}C'Q$ is
        $$\begin{pmatrix}
        c'_{11}     & c'_{12} & \cdots     & c'_{1\ n-1} & c'_{1k}+c'_{1n} \\
        \vdots \\
        c'_{k-11}     & c'_{k-1 2}           & \cdots     & c'_{k-1 n-1} & c'_{k-1k}+c'_{k-1n} \\
        c'_{k1}-c'_{n1} & c'_{k2}-c'_{n2} &\cdots      &c'_{kn-1}-c'_{nk-1}& c'_{kk}+c'_{kn}-c'_{nk}-x'_{nn}\\
        c'_{k+11}     & c'_{k+12}           & \cdots     & c'_{k+1\ n-1} & c'_{k+1k}+c'_{k+1n} \\
        \vdots              & \vdots & \ddots              & \vdots & \vdots \\
        c'_{n1}             & c'_{n2} & \cdots             & c'_{n\ n-1} & c'_{nk}+c'_{nn} \\
        \end{pmatrix}.$$
    \end{proof}
\end{frame}

\begin{frame}
    \frametitle{{Continue}}
    \begin{proof}
        (ii)
            By Lemma~\ref{v_rooted} $v'=Qu$ is $k$-rooted.  
            Since $Q^{-1}C'Qu=\rho(C')u$ by the assumption, we have
            $Q^{-1} C' Q u  = Q^{-1} \rho(C') Qu  =\rho(C')u$  \\
            $C'Qu=\rho(C')Qu$.

        (iii)
        Since $C'$ and $Q^{-1}C'Q$ have the same set of eigenvalues, clearly $\rho(C')$ = $\rho(Q^{-1}C'Q)$.

    \end{proof}
\end{frame}

% \begin{frame}

\begin{lem}\label{l_diag}   %end of label
If a square matrix $C'$ has a rooted eigenvector for $\lambda'$, then $C'+dI$ also has
the same rooted eigenvector for $\lambda'+d,$ where $d$ is a constant and $I$ is the identity matrix with the same size of $C'$.
\end{lem}

% \end{frame}

\begin{frame}

\begin{thm}
    Let $C$ be an $n\times n$ nonnegative Tmatrix. For $1\leq i \leq n$ and $1\leq j\leq n-1$, choose $c'_{ij}$
    such that $c'_{ij}\geq c_{ij}$ and $c'_{kj}\geq c'_{nj}>0$, and choose $r'_i$ such that $r'_i\geq c_{ik}+c_{in}$, and
    $r'_k \geq r'_n$. Moreover choose $c'_{in}:=r'_i-c'_{ik}$. Then $\rho(C)\leq \rho(C')$, when $C'=(c'_{ij})$.
\end{thm}

\end{frame}

\begin{frame}
    \frametitle{ Proof}

The assumptions are necessary that $PCQ \leq PC'Q$, and C' is k-rooted, by (~\ref{v_rooted}), For certain d, if C'+d*I is k-rooted, then it has a k-rooted eigenvector with its spectral radius $\lambda + d$. C' would share the same eigenvector with C'+d*I and has eigenvalue $\lambda$. So C'+d*I and C+d*I meet the conditions of (\ref{l_diag}), and we can show that $\rho(C' + d*I) \geq \rho(C +d*I)$ and then $\rho(C') \geq \rho(C)$  \qed    

\end{frame}

\begin{frame}
    \frametitle{ Example}
    For the following $4\times 4$ matrix
    $$C=\begin{pmatrix}
    0 & 0 & 1 & 1\\
    1 & 0 & 0 & 1\\
    1 & 1 & 0 & 1\\
    1 & 1 & 1 & 0
    \end{pmatrix},$$
    we choose
    $$C'=\begin{pmatrix}
    0 & 0 & 1 & 1\\
    1 & 0 & 1 &  0\\
    1 & 1 & 0 & 1\\
    1 & 1 & 1 & 0
    \end{pmatrix}.$$
    Then
    $\rho(C)\leq \rho(C')$ by previous theorem.
\end{frame}

\begin{frame}
    \frametitle{Counterexample}
    For the following two $4\times 4$ matrices
    $$C=\begin{pmatrix}
    0 & 0 & 1 & 1\\
    1 & 0 & 0 & 1\\
    1 & 1 & 0 & 0\\
    1 & 1 & 1 & 0
    \end{pmatrix},\quad C'=\begin{pmatrix}
    0 & 0 & 1 & 1\\
    1 & 0 & 1 &  0\\
    1 & 1 & 0 & 0\\
    1 & 1 & 1 & 0
    \end{pmatrix},$$ 
    we have $CQ\leq C'Q$, but 
    $\rho(C)=2.234\not\leq 2.148= \rho(C')$. 
    This is because $c'_{33}+c'_{34}\not\geq c'_{43}+c'_{44}$. 
\end{frame}

\begin{frame}
\begin{thebibliography}{20}
\normalsize
\addcontentsline{toc}{chapter}{Bibliography}

\bibitem{chang}
Yen-Jen Cheng, {\it  A matrix realization of spectral bounds
of the spectral radius of a nonnegative matrix}, Ph.D. Thesis, NCTU, 2018.

\bibitem{prn_fros1}
A. E. Brouwer, W. H. Haemers, {\it Spectra of graphs}, Springer, 2012

\bibitem{prn_fros2}
R. A. Horn , C. R. Johnson, {\it Matrix analysis}, Cambrigde University Press, 1985.


\end{thebibliography}
\end{frame}


\end{document}
