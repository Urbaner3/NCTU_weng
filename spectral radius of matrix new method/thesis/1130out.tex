\documentclass[12pt]{report}%{article}

\topmargin 0pt

\topmargin=-1.5cm
\oddsidemargin=0.7cm
\textheight=23.5cm
\textwidth=15cm
\setlength{\baselineskip}{24pt}
\renewcommand{\baselinestretch}{1.5} %行




\usepackage{fontspec}   %加這個就可以設定字體
\usepackage{xeCJK}       %讓中英文字體分開設置
%\usepackage{times}
\usepackage{amssymb}
\usepackage{amsmath}
\usepackage{amsthm}
\usepackage{ulem}
\usepackage{amsfonts}
\usepackage{mathrsfs}
\usepackage{tabularx,array}
\usepackage{pgf,tikz}
%\usepackage{colortbl}

\usetikzlibrary{arrows}
\setCJKmainfont{標楷體} %設定中文為系統上的字型,而英文不去更動,使用原TeX字型
\XeTeXlinebreaklocale "zh"             %這兩行一定要加,中文才能自動換行
\XeTeXlinebreakskip = 0pt plus1pt     %這兩行一定要加,中文才能自動換行
\title{Combinatorial Identities from Lagrange's Interpolation Polynomial}
\author{Student: Yen-Jung Huang  ~~~~~~~~~~~~~~~~~~~~~~~~~~Advisor: Chih-Wen Weng}
\date{} %不要日期

\def\UrlFont{\rm}


\theoremstyle{plain}
\newtheorem{thm}{Theorem}[chapter]
\newtheorem{cor}[thm]{Corollary}
\newtheorem{lem}[thm]{Lemma}
\newtheorem{prop}[thm]{Proposition}
\newtheorem{remark}[thm]{Remark}
\newtheorem{eg}[thm]{Example}
\newtheorem{conj}[thm]{Conjecture}


\theoremstyle{definition}
\newtheorem{defn}[thm]{Definition}
\newtheorem{ex}[thm]{Exercise}
\newtheorem{prob}[thm]{Problem}
\newtheorem{exam}[thm]{Example}
\newtheorem{nota}[thm]{Notation}
\newtheorem{rem}[thm]{Remark}
\newtheorem{ques}[thm]{Question}
\newtheorem{pof}[thm]{Proof.}

%\renewcommand {\refname} {Bibliography}

\begin{document}

\chapter{Preliminaries}

Let $\mathbb{R}$ and $\mathbb{C}$ denote the field of real numbers and complex numbers respectively.


\begin{defn}
    Let $C$ be an $n \times n$ real matrix, and $u \in \mathbb{R}^n$ be a column vector. The scalar $\lambda \in \mathbb{C}$ is an $\it{eigenvalue}$ of C corresponding to an $\it{eigenvector}$ if $Cu = \lambda u$ 
   
\end{defn}

\begin{defn}

When $C$ is a real square matrix, the $\textit {spectral radius} $ $\rho(C)$ of $C$ is defined by
$$\rho(C):=\max\{~|\lambda|~\ |~~\lambda\hbox{ is an eigenvalue of $C$}\},$$
where $|\lambda|$ is the magnitude of complex number $\lambda.$
\end{defn}

We are interested in spectral radius of the following matrix associated with a simple graph.

\begin{defn} Given an undirected graph
G, the$\textit{ adjacency matrix}$ of G is the square matrix $A = (a_{ij})$ indexed by vertices of G,
and
\[a_{ij} =\begin{cases}
1, \text{if $i$ is adjacent to $j$}, \\
0, \text{otherwise.}
\end{cases}
\]

\end{defn}

\begin{defn}
Given an undirected graph G, the $\textit{spectral radius}$ of G is the spectral radius of the adjacency matrix A.
\end{defn}



    We introduce a notation of submatrix, which is taken from some columns and some rows of a matrix.

 
\begin{defn}For a matrix $C=(c_{ij})$ and subsets $\alpha$, $\beta$ of row indices and column indices of $C$ respectively,  We use $C[\alpha|\beta]$ to denote the submatrix of $C$ with size $ |\alpha| \times |\beta| $ that has entries $c_{ij}$ for $i\in \alpha$ and $j\in\beta$,
\begin{comment}
$C[\alpha|\beta):=C[\alpha|\overline{\beta}],$ where $\overline{\beta}$ is the complement of $\beta$ in the set of column indices, and
similarly, for the definitions of $C(\alpha|\beta]$ and $C(\alpha|\beta).$
 For $\ell\in \mathbb{N},$ $[\ell]:=\{1, 2, \ldots, \ell\},$ symbol $-$ is the complete set of indices, and we use  $i$ to denote the singleton subset $\{i\}$ to reduce the double use of parentheses. For example of the $n\times n$ matrix $C$,
 $C[-|n)=C[[n]|[n-1]]$ is the $n\times (n-1)$ submatrix of $C$ obtained by deleting the last column of $C$.

    The bracket [] and parentheses () can be used together is the notation of sub-matrix, such as $C[k|e)$ or $C(k|e]$.
\end{comment}

\end{defn}

\begin{thm}
        If $C$ is a nonnegative square matrix, then spectral radius $\rho(C)$ is an eigenvalue of $C$ with a corresponding nonnegative right eigenvector and a corresponding nonnegative left eigenvector.
\end{thm}


The following theorem is from \cite{chang}.

\begin{thm}
 Let $C=(c_{ij})$, $C'=(c'_{ij})$, $P$ and $Q$ be  $n\times n$ matrices.
Assume that
\begin{enumerate}
\item[(i)]    $PCQ\leq PC'Q$;
\item[(ii)]  there exist a nonnegative column vector $u=(u_1, u_2, \ldots, u_n)^T$  and a scalar $\lambda'\in \mathbb{R}$ such that $\lambda'$ is an eigenvalue of $C'$ with associated eigenvector $Qu$;
\item[(iii)] there exist a nonnegative row vector $v^T=(v_1, v_2, \ldots, v_n)$  and a scalar $\lambda\in \mathbb{R}$
such that $\lambda$ is an eigenvalue of $C$ with associated  left eigenvector $v^TP$; and
\item[(iv)] $v^TPQu>0.$
\end{enumerate}
 Then $\lambda\leq \lambda'$.
    Moreover, $\lambda=\lambda'$
if and only if
    \begin{equation*}
        \label{e3}
(PC'Q)_{ij}=(PCQ)_{ij}\qquad \hbox{for~}1\leq i, j\leq n \hbox{~with~} v_i\ne 0 \hbox{~and~} u_j\ne 0.
\end{equation*}
\end{thm}


\begin{proof}
            Multiplying the nonnegative vector $u$ in (ii) to the right of both terms of  (i),
        \begin{equation}
            \label{e1}
            PCQu\leq PC'Qu=\lambda'PQu.
        \end{equation}
        Multiplying the nonnegative left eigenvector $v^T$ of $C$ for $\lambda$ in assumption (iii) to the left of all terms  in (\ref{e1}), we have
        \begin{equation}
            \label{e2}
            \lambda v^TPQu=v^TPCQu\leq v^TPC'Qu=\lambda' v^TPQu.
        \end{equation}
        Now delete the positive term $v^TPQu$ by assumption (iv) to obtain $\lambda\leq \lambda'$ and finish the proof of the first part.



        Assume that $\lambda=\lambda'$, so the inequality in (\ref{e2}) is an equality.  Especially $(PCQu)_i=(PC'Qu)_i$ for any $i$ with $v_i\not=0.$ Hence, $(PCQ)_{ij}=(PC'Q)_{ij}$ for any $i$ with $v_i\not=0$ and any $j$ with $u_j\not=0.$
         Conversely, (\ref{e3}) implies $$v^TPCQu=\sum_{i,j} v_i(PCQ)_{ij}u_j=\sum_{i,j} v_i(PC'Q)_{ij}u_j=v^TPC'Qu,$$ so
            $\lambda=\lambda'$ by (\ref{e2}). \qed

 \end{proof}
 
 
 
 \newpage
\begin{thebibliography}{20}
\normalsize
\addcontentsline{toc}{chapter}{Bibliography}

\bibitem{chang}
Yen-Jen Chenga, Chih-wen Wenga, A matrix realization of spectral bounds
of the spectral radius of a nonnegative matrix, (journal),....

\bibitem{catalan}
\newblock{James Haglund, The $q$,$t$-Catalan Numbers and the Space of Diagonal Harmonics: With an Appendix on the Combinatorics of Macdonald Polynomials},
\newblock{\it University Lectures Series}, vol. 41, 2008

\bibitem{combinatorics}
\newblock{J.H. van Lint and R.M. Wilson, A Course in Combinatorics, second edition},
\newblock{\it} Cambridge University Press, New York, 2001.

\bibitem{euler}
\newblock{H. W. Gould, Euler's formula nth differences of powers},
\newblock{\it The American Mathematical Monthly }, vol. 85, No. 6, 450-467,1978

\bibitem{rook}
\newblock{Feryal Alayomt and Nicholas Krzywonos, Rook polynomials in three and higher dimensions},
\newblock{\it }, 2009

\bibitem{brualdi}
\newblock{Richard A. Brualdi, Introductory Combinatorics, fifth edition},
\newblock{\it },  Prentice Hall, New Jersey, 2009.

\bibitem{qanalogue}
\newblock{Bruce E. Sagan, Congruence Properties of $q$-Analogs},
\newblock{\it Advances in Mathematics}, 95, 127-143,1992

\bibitem{lagrange}
\newblock{Eric W. Weisstein, Lagrange Interpolating Polynomial}, \newblock{MathWorld}

\bibitem{9}
\newblock{郭子翔,一個新恆等式的發現與研究}, preprint.

\end{thebibliography}


\end{document}