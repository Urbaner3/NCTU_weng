\documentclass[12pt]{report}%{article}

\topmargin 0pt

\topmargin=-1.5cm
\oddsidemargin=0.7cm
\textheight=23.5cm
\textwidth=15cm
\setlength{\baselineskip}{24pt}
\renewcommand{\baselinestretch}{1.5} %行距




\usepackage{fontspec}   %加這個就可以設定字體
\usepackage{xeCJK}       %讓中英文字體分開設置
%\usepackage{times}
\usepackage{amssymb}
\usepackage{amsmath}
\usepackage{amsthm}
\usepackage{ulem}
\usepackage{amsfonts}
\usepackage{mathrsfs}
\usepackage{tabularx,array}
\usepackage{pgf,tikz}
%\usepackage{colortbl}

\usetikzlibrary{arrows}
\setCJKmainfont{標楷體} %設定中文為系統上的字型,而英文不去更動,使用原TeX字型
\XeTeXlinebreaklocale "zh"             %這兩行一定要加,中文才能自動換行
\XeTeXlinebreakskip = 0pt plus1pt     %這兩行一定要加,中文才能自動換行
\title{Combinatorial Identities from Lagrange's Interpolation Polynomial}
\author{Student: Yen-Jung Huang  ~~~~~~~~~~~~~~~~~~~~~~~~~~Advisor: Chih-Wen Weng}
\date{} %不要日期

\def\UrlFont{\rm}


\theoremstyle{plain}
\newtheorem{thm}{Theorem}[chapter]
\newtheorem{cor}[thm]{Corollary}
\newtheorem{lem}[thm]{Lemma}
\newtheorem{prop}[thm]{Proposition}
\newtheorem{remark}[thm]{Remark}
\newtheorem{eg}[thm]{Example}
\newtheorem{conj}[thm]{Conjecture}


\theoremstyle{definition}
\newtheorem{defn}[thm]{Definition}
\newtheorem{ex}[thm]{Exercise}
\newtheorem{prob}[thm]{Problem}
\newtheorem{exam}[thm]{Example}
\newtheorem{nota}[thm]{Notation}
\newtheorem{rem}[thm]{Remark}
\newtheorem{ques}[thm]{Question}
\newtheorem{pof}[thm]{Proof.}

%\renewcommand {\refname} {Bibliography}


\begin{document}
%封面

\thispagestyle{empty}
\begin{center}
{ \Huge 國~~~~立~~~~交~~~~通~~~~大~~~~學}~\\~\\

\bigskip

{ \Huge 應~用~數~學~系}~\\~\\

\bigskip

{ \Huge 碩~~士~~論~~文}~\\~\\

\bigskip \bigskip\bigskip\bigskip\bigskip\bigskip

{ \Huge TBA}~\\~\\

\bigskip

{ \Huge 待輸入}~\\~\\
~\\~\\~\\~\\~\\
\bigskip \bigskip\bigskip\bigskip\bigskip\bigskip
\bigskip \bigskip\bigskip\bigskip\bigskip\bigskip
\bigskip\bigskip\bigskip

{ \Large
\begin{tabular}{rcl}
研究生&:&XXX\\
指導教授&:&翁志文~教授
\end{tabular} }

\bigskip\bigskip
{ \Large 中~~華~~民~~國~~一~~百~~零~~八~~年~~一~~月 }
\large
\end{center}
\pagebreak


%%%%%%%%%%%%%%%%%%%%%%%%%%%%%%%%%%%%%%%%%%%%%%%%%%%%%%%%%%%%%%%%%%%%%%%%%
\renewcommand{\baselinestretch}{2} %行距
\thispagestyle{empty}
\begin{center}
{
\Large
TBA\\
待輸入 \\~\\
\begin{tabular}{lccr}
Student: XXX  &&~~~& Advisor: Chih-Wen Weng\\
研究生:陳科翰  &&~~~& 指導教授:翁志文~教授
\end{tabular}
}~\\

\bigskip

\renewcommand{\baselinestretch}{1} %行距

{ \LARGE 國~~~~立~~~~交~~~~通~~~~大~~~~學}\\~\\
{ \LARGE 應~用~數~學~系}\\~\\
{ \LARGE 碩~~士~~論~~文}\\~\\~\\~\\~\\
\renewcommand{\baselinestretch}{1} %行距
{ \large A Thesis

Submitted to Department of Applied Mathematics

College of Science

National Chiao Tung University

in Partial Fulfillment of Requirements

for the Degree of Master

in Applied Mathematics
\bigskip \medskip

January 2019

Hsinchu, Taiwan, Republic of China \bigskip \medskip

 中~~華~~民~~國~~一~~百~~零~~八~~年~~一~~月 }
\end{center}
\pagebreak
%中文摘要
\pagenumbering{roman}
\begin{center}
{  \LARGE
待輸入
\bigskip\bigskip\bigskip

研究生:陳科翰  ~~~~~~~~~~ 指導教授:翁志文~教授 \\
國立交通大學  \\
\bigskip
應用數學系
\bigskip\bigskip\bigskip\bigskip
} \\~\\~\\~\\
\addcontentsline{toc}{chapter}{Abstract (in Chinese)}
{\large 摘~要}
\end{center}
 \bigskip

待輸入\\
\bigskip

\noindent 關鍵詞:待輸入。
\pagebreak



%英文摘要
\begin{center}{\LARGE
TBA
\bigskip\bigskip\bigskip}

{ \large
Student: XXX  ~~~~~ Advisor: Chih-Wen Weng \\
\Large

Department ~of~ Applied ~Mathematics
\bigskip

National~ Chiao ~Tung~ University
\bigskip\bigskip\bigskip\bigskip}\\
{\large Abstract}
\end{center}

%\begin{abstract}
\addcontentsline{toc}{chapter}{Abstract (in English)}

TBA
\bigskip


\noindent {\bf Keywords}: TBA
%\end{abstract}
\pagebreak
\renewcommand{\baselinestretch}{1.2}
% 目錄
\large
\tableofcontents


\pagebreak
% Introduction

\chapter{Introduction}
\normalsize
\pagenumbering{arabic}






\chapter{Preliminaries}

\begin{defn}
Spectral radius of a matrix $C$ is the largest absolute value of real eigenvalue of $C$. And here we are interested in spectral radius of the adjacency matrix for some simple graph.
\end{defn}

\begin{defn}
    Eigenvalue of a matrix $C$ corresponds to an eigenvector $u$, where $u$ and its eigenvalue $\lambda $  satisfies $Cu = \lambda u$.
\end{defn}


    We introduce a notation of sub-matrix, which is taken from some columns and some rows of a matrix.

    \begin{defn}We use $C[\alpha|\beta]$ to denote the sub-matrix of rows $\alpha$ of $C$ and columns $\beta$ of $C$ and $C(\alpha|\beta)$ is the sub-matrix of rows $[n]$ - $\alpha$ of $C$ and columns $[n] - \beta$ of $C$.
    The bracket [] and parentheses () can be used together is the notation of sub-matrix, such as $C[k|e)$ or $C(k|e]$.
\end{defn}

\begin{thm}
        If $C$ is nonnegative square matrix, then the following holds. \\
        The spectral radius $\rho(C)$ is an eigenvalue of $C$ with a corresponding nonnegative right eigenvector and a corresponding nonnegative left eigenvector.
\end{thm}


The following theorem is from [].

\begin{thm}
 Let $C=(c_{ij})$, $C'=(c'_{ij})$, $P$ and $Q$ be  $n\times n$ matrices.
Assume that
\begin{enumerate}
\item[(i)]    $PCQ\leq PC'Q$;
\item[(ii)]  there exist a nonnegative column vector $u=(u_1, u_2, \ldots, u_n)^T$  and a scalar $\lambda'\in \mathbb{R}$ such that $\lambda'$ is an eigenvalue of $C'$ with associated eigenvector $Qu$;
\item[(iii)] there exist a nonnegative row vector $v^T=(v_1, v_2, \ldots, v_n)$  and a scalar $\lambda\in \mathbb{R}$
such that $\lambda$ is an eigenvalue of $C$ with associated  left eigenvector $v^TP$; and
\item[(iv)] $v^TPQu>0.$
\end{enumerate}
 Then $\lambda\leq \lambda'$.
    Moreover, $\lambda=\lambda'$
if and only if
    \begin{equation*}
        \label{e3}
(PC'Q)_{ij}=(PCQ)_{ij}\qquad \hbox{for~}1\leq i, j\leq n \hbox{~with~} v_i\ne 0 \hbox{~and~} u_j\ne 0.
\end{equation*}
\end{thm}


\begin{proof}
            Multiplying the nonnegative vector $u$ in (ii) to the right of both terms of  (i),
        \begin{equation}
            \label{e1}
            PCQu\leq PC'Qu=\lambda'PQu.
        \end{equation}
        Multiplying the nonnegative left eigenvector $v^T$ of $C$ for $\lambda$ in assumption (iii) to the left of all terms  in (\ref{e1}), we have
        \begin{equation}
            \label{e2}
            \lambda v^TPQu=v^TPCQu\leq v^TPC'Qu=\lambda' v^TPQu.
        \end{equation}
        Now delete the positive term $v^TPQu$ by assumption (iv) to obtain $\lambda\leq \lambda'$ and finish the proof of the first part.



        Assume that $\lambda=\lambda'$, so the inequality in (\ref{e2}) is an equality.  Especially $(PCQu)_i=(PC'Qu)_i$ for any $i$ with $v_i\not=0.$ Hence, $(PCQ)_{ij}=(PC'Q)_{ij}$ for any $i$ with $v_i\not=0$ and any $j$ with $u_j\not=0.$
         Conversely, (\ref{e3}) implies $$v^TPCQu=\sum_{i,j} v_i(PCQ)_{ij}u_j=\sum_{i,j} v_i(PC'Q)_{ij}u_j=v^TPC'Qu,$$ so
            $\lambda=\lambda'$ by (\ref{e2}). \qed

 \end{proof}

\chapter{Our result}

 Throughout fix $k\in [n-1]$. Let $E_{kn}$ denote the $n\times n$ binary matrix with a unique $1$ appearing in the  position $k,n$ of $E_{kn}$. We will apply the previous theorem with $P=I$ and $$Q=I+E_{kn}=\begin{pmatrix}
1 &  & & &  & 0 \\
 & 1 &  &      &  &  \\
 &  & \ddots & &  & 1 \\
 &  &        & &  &  \\
  &  & & & 1 &  \\
0 &  & & &  & 1 \\
\end{pmatrix}.$$
\begin{defn}\label{v_rooted}
 ~A column vector $v'=(v'_1,v'_2,\ldots,v'_n)^T$ is called {\it $k$-rooted}  if $v'_{j} \geq 0$ for $1 \leq  j \leq n$ and $v'_k\geq v'_n.$
\end{defn}
\bigskip

The following Lemma is immediate from the above definition.
\bigskip

\begin{lem}
If ~$u=(u_1, u_2, \ldots, u_n)^T$ and $v'=(v'_1, v'_2, \ldots, v'_n):=Qu=(u_1,\ldots, u_{k-1},u_k+u_n, u_{k+1}, \ldots,  u_n)^T$, then
\begin{enumerate}
\item[(i)] $v'$ is $k$-rooted  if and only if  $u$ is nonnegative;
\item[(ii)] $u_k>0$ if and only if $v'_k>v'_n$.
\end{enumerate}
\qed
\end{lem}



\begin{thm} \label{thm_main}
    Let $C=(c_{ij})$, $C'=(c'_{ij})$ be  $n\times n$ matrices.
Assume that
\begin{enumerate}
\item[(i)]   $C[-|n)\leq C'[-|n)$ and $c_{ik}+c_{in}\leq c'_{ik}+c'_{in}$ for all $1\leq i\leq n$;
\item[(ii)] there exist a $k$-rooted vector $v'=(v'_1, v'_2, \ldots, v'_n)^T$ and a scalar $\lambda'\in \mathbb{R}$
such that $\lambda'$ is an eigenvalue of $C'$ with associated eigenvector $v'$;
\item[(iii)] there exists a nonnegative vector $v^T=(v_1, v_2, \ldots, v_n)$ and a scalar $\lambda\in \mathbb{R}$ such that $\lambda$ is an eigenvalue of $C$ with associated left eigenvector $v^T$;
\item[(iv)] $v^Tv'>0.$
\end{enumerate}
 Then $\lambda\leq \lambda'$.
Moreover, $\lambda=\lambda'$
if and only if
\begin{enumerate}
\item[(a)] $c_{ik}+c_{in}=c'_{ik}+c'_{in}$\qquad for $1\leq i\leq n$ with $v_i\not=0$ and $v'_n\not=0;$
\item[(b)]
$c'_{ij}=c_{ij}\qquad \hbox{for~}1\leq i\leq n,~1\leq j\leq n-1 \hbox{~with~} v_i\ne 0 \hbox{~and~} v'_j> v'_n.$
\end{enumerate} \qed
\end{thm}


\begin{defn}
For a matrix $C'=(c'_{ij})$ of $n$ columns, the $(k, n)$-{\it sum} vector of $C'$ is the vector of the sum of the $k$-th and  $n$-th columns of $C'$, where $k\leq n-1$.
\end{defn}

\begin{defn}\label{m_rooted}
A  matrix $C'=(c'_{ij})$ is called {\it $k$-rooted}  if its  columns and its $(k, n)$-sum vector are all $k$-rooted except the last column of $C'$.
\end{defn}

\begin{thm}
Let $C'=(c'_{ij})$ be an $n\times n$ nonnegative matrix. Then the following (i)-(iii) hold.
    \begin{enumerate}
        \item[(i)]$C'$ is a $k$-rooted matrix, if and only if, $Q^{-1}C'Q$ is nonnegative.
        \item[(ii)]$C'$ has a $k$-rooted eigenvector $v'=Qu$ for $\rho(C')$ as an eigenvalue. $u$ is a nonnegative eigenvector of $Q^{-1}C'Q$ for $\rho(C')$.
        \item[(iii)] $\rho(C')$ = $\rho(Q^{-1}C'Q)$
    \end{enumerate}
\end{thm}

\begin{proof}
(i) is immediate from Definition~\ref{m_rooted} and the observation that
$$Q^{-1}=I-E_{kn}=\begin{pmatrix}
1 &  & & &  & 0 \\
 & 1 &  &      &  &  \\
 &  & \ddots & &  & -1 \\
 &  &        & &  &  \\
  &  & & & 1 &  \\
0 &  & & &  & 1 \\
\end{pmatrix},$$
and $Q^{-1}C'Q$ is
$$\begin{pmatrix}
c'_{11}     & c'_{12} & \cdots     & c'_{1\ n-1} & c'_{1k}+c'_{1n} \\
\vdots \\
c'_{k-11}     & c'_{k-1 2}           & \cdots     & c'_{k-1 n-1} & c'_{k-1k}+c'_{k-1n} \\
c'_{k1}-c'_{n1} & c'_{k2}-c'_{n2} &\cdots      &c'_{kn-1}-c'_{nk-1}& c'_{kk}+c'_{kn}-c'_{nk}-x'_{nn}\\
c'_{k+11}     & c'_{k+12}           & \cdots     & c'_{k+1\ n-1} & c'_{k+1k}+c'_{k+1n} \\
\vdots              & \vdots & \ddots              & \vdots & \vdots \\
c'_{n1}             & c'_{n2} & \cdots             & c'_{n\ n-1} & c'_{nk}+c'_{nn} \\
\end{pmatrix}.
$$



(ii)
By Lemma~\ref{v_rooted} $v'=Qu$ is $k$-rooted.
Since $Q^{-1}C'Qu=\rho(C')u$ by the assumption, we have
$C'Qu=\rho(C')Qu$.



(iii)
Since $C'$ and $Q^{-1}C'Q$ have the same set of eigenvalues, clearly $\rho(C')$ = $\rho(Q^{-1}C'Q)$.

\end{proof}


\begin{lem}\label{l_diag}
If a square matrix $C'$ has a rooted eigenvector for $\lambda'$, then $C'+dI$ also has
the same rooted eigenvector for $\lambda'+d,$ where $d$ is a constant and $I$ is the identity matrix with the same size of $C'$.
\end{lem}

\chapter{Conclusion}

\begin{thm}
Let $C$ be an $n\times n$ nonnegative matrix. For $1\leq i \leq n$ and $1\leq j\leq n-1$, choose $c'_{ij}$
such that $c'_{ij}\geq c_{ij}$, and then choose $r'_i$ such that $r'_i\geq c_{ik}+c_{in}$ and $c'_{kj}\geq c'_{nj}>0$, and
let $c'_{in}:=r'_i-c'_{ik}$ with $r'_k \geq r'_n$. Whereas $c'_{kk}\geq c'_{nk}$ is not necessary. Then the $n\times n$ matrix $C'=(c'_{ij})$ has a positive $k$-rooted eigenvector for $\rho(C')$ and $\rho(C)\leq \rho(C')$.
\end{thm}



{\bf Proof}

The assumptions are necessary that $PCQ \leq PC'Q$, and C' is k-rooted, by \ref{l_diag}, For certain d, if C'+d*I is k-rooted, then it has a k-rooted eigenvector with its spectral radius $\lambda + d$. C' would share the same eigenvector with C'+d*I and has eigenvalue $\lambda$. So C'+d*I and C+d*I meet the conditions of \ref{thm_main}, and we can show that $\rho(C' + d*I) \geq \rho(C +d*I)$ and then $\rho(C') \geq \rho(C)$  \qed



{Example}
For the following $4\times 4$ matrix
$$C=\begin{pmatrix}
0 & 0 & 1 & 1\\
1 & 0 & 0 & 1\\
1 & 1 & 0 & 1\\
1 & 1 & 1 & 0
\end{pmatrix},$$
we choose
$$C'=\begin{pmatrix}
0 & 0 & 1 & 1\\
1 & 0 & 1 &  0\\
1 & 1 & 0 & 1\\
1 & 1 & 1 & 0
\end{pmatrix}.$$
Then
$\rho(C)\leq \rho(C')$ by previous theorem.




{\bf Counterexample}
For the following two $4\times 4$ matrices
$$C=\begin{pmatrix}
0 & 0 & 1 & 1\\
1 & 0 & 0 & 1\\
1 & 1 & 0 & 0\\
1 & 1 & 1 & 0
\end{pmatrix},\quad C'=\begin{pmatrix}
0 & 0 & 1 & 1\\
1 & 0 & 1 &  0\\
1 & 1 & 0 & 0\\
1 & 1 & 1 & 0
\end{pmatrix},$$
we have $CQ\leq C'Q$, but
$\rho(C)=2.234\not\leq 2.148= \rho(C')$.
This is because $c'_{33}+c'_{34}\not\geq c'_{43}+c'_{44}$.






\newpage
\begin{thebibliography}{20}
\normalsize
\addcontentsline{toc}{chapter}{Bibliography}
\bibitem{catalan}
\newblock{James Haglund, The $q$,$t$-Catalan Numbers and the Space of Diagonal Harmonics: With an Appendix on the Combinatorics of Macdonald Polynomials},
\newblock{\it University Lectures Series}, vol. 41, 2008

\bibitem{combinatorics}
\newblock{J.H. van Lint and R.M. Wilson, A Course in Combinatorics, second edition},
\newblock{\it} Cambridge University Press, New York, 2001.

\bibitem{euler}
\newblock{H. W. Gould, Euler's formula nth differences of powers},
\newblock{\it The American Mathematical Monthly }, vol. 85, No. 6, 450-467,1978

\bibitem{rook}
\newblock{Feryal Alayomt and Nicholas Krzywonos, Rook polynomials in three and higher dimensions},
\newblock{\it }, 2009

\bibitem{brualdi}
\newblock{Richard A. Brualdi, Introductory Combinatorics, fifth edition},
\newblock{\it },  Prentice Hall, New Jersey, 2009.

\bibitem{qanalogue}
\newblock{Bruce E. Sagan, Congruence Properties of $q$-Analogs},
\newblock{\it Advances in Mathematics}, 95, 127-143,1992

\bibitem{lagrange}
\newblock{Eric W. Weisstein, Lagrange Interpolating Polynomial}, \newblock{MathWorld}

\bibitem{9}
\newblock{郭子翔,一個新恆等式的發現與研究}, preprint.

\end{thebibliography}


\end{document}