\documentclass{article}

% \topmargin 0pt

% \topmargin=-1.5cm
% \oddsidemargin=0.7cm
% \textheight=23.5cm
% \textwidth=15cm
% \setlength{\baselineskip}{24pt}
% \renewcommand{\baselinestretch}{1.5} %行距




\usepackage{fontspec}   %加這個就可以設定字體
\usepackage{xeCJK}       %讓中英文字體分開設置
%\usepackage{times}
\usepackage{amssymb}
\usepackage{amsmath}
\usepackage{amsthm}
\usepackage{ulem}
% \usepackage{amsfonts}
\usepackage{mathrsfs}
% \usepackage{tabularx,array}
% \usepackage{pgf,tikz}
% \usepackage{blkarray} %line 80-92 Q formula matrix index
%\usepackage{colortbl}
\usepackage{enumitem}% enumerate labels   roman
% \usetikzlibrary{arrows}
\setCJKmainfont{標楷體} %設定中文為系統上的字型,而英文不去更動,使用原TeX字型
% \XeTeXlinebreaklocale "zh"             %這兩行一定要加,中文才能自動換行
% \XeTeXlinebreakskip = 0pt plus1pt     %這兩行一定要加,中文才能自動換行
% \title{Combinatorial Identities from Lagrange's Interpolation Polynomial}
% \author{Student: Yen-Jung Huang  ~~~~~~~~~~~~~~~~~~~~~~~~~~Advisor: Chih-Wen Weng}
% \date{} %不要日期

\def\UrlFont{\rm}


\theoremstyle{plain}
\newtheorem{thm}{Theorem}[subsection]
\newtheorem{cor}[thm]{Corollary}
\newtheorem{lem}[thm]{Lemma}
\newtheorem{prop}[thm]{Proposition}
\newtheorem{remark}[thm]{Remark}
\newtheorem{eg}[thm]{Example}
\newtheorem{conj}[thm]{Conjecture}


\theoremstyle{definition}
\newtheorem{defn}[thm]{Definition}
\newtheorem{ex}[thm]{Exercise}
\newtheorem{prob}[thm]{Problem}
\newtheorem{exam}[thm]{Example}
\newtheorem{nota}[thm]{Notation}
\newtheorem{rem}[thm]{Remark}
\newtheorem{ques}[thm]{Question}
\newtheorem{pof}[thm]{Proof.}

\usepackage{comment}

%\renewcommand {\refname} {Bibliography}



\begin{document}

\section{Preliminaries}
    
Let $\mathbb{R}$ and $\mathbb{C}$ denote the
 field of real numbers and complex numbers respectively.



\begin{defn}       
Let $C$ be an $n \times n$ real nonnegative matrix, and $u \in \mathbb{R}^n$ be a
 nonzero column vector. The scalar $\lambda \in \mathbb{C}$ is an $\it{eigenvalue}$ 
of C corresponding to the $\it{eigenvector}$ $u,$  if $Cu = \lambda u.$
        

%Also, $C$ has an left eigenvector $v^T$, when $v^TC=\lambda v^T$.
\end{defn}


\begin{defn}
    When $C$ is an $n \times n$ real matrix, the $\textit {spectral radius} $ $\rho(C)$
        of $C$ is defined by 
        $$\rho(C):=\max\{~|\lambda|~\ |~~\lambda\hbox{ is an eigenvalue of $C$}\},$$
    where $|\lambda|$ is the magnitude of complex number $\lambda.$
\end{defn}

    

\begin{thm}[Perron–Frobenius theorem]\cite{chang}
    If $C$ is nonnegative square matrix, then the spectral radius $\rho(C)$ is an
    eigenvalue of $C$ with a corresponding nonnogative right eigenvector and a
    corresponding nonnegative left eigenvector.
\end{thm}

We are interested in spectral radius of the following matrix associated with a simple graph.

\begin{defn} 
    Given an undirected graph G, the$\textit{ adjacency matrix}$ of G is the square 
    matrix $A = (a_{ij})$ indexed by vertices of G, and
     \[a_{ij} =\begin{cases} 
        1, \text{if $i$ is adjacent to $j$}, \\
        0, \text{otherwise.}
            \end{cases}
     \]
\end{defn}
    
\begin{defn}
Given an undirected graph G, the $\textit{spectral radius}$  $\rho(G) $ of G is the spectral
 radius of the adjacency matrix of G.
\end{defn}

    We introduce a notation of submatrix, which is taken from some columns and some rows of a
     matrix.

\begin{defn}
    For a matrix $C=(c_{ij})$ and subsets $\alpha$, $\beta$ of row indices and column 
    indices of $C$ respectively,  We use $C[\alpha|\beta]$ to denote the 
    submatrix of $C$ with size $ |\alpha| \times |\beta| $ that has entries $c_{ij}$ for $i\in \alpha$
    and $j\in\beta$,
\end{defn}


We introduce two matrices $P$ and $Q$ in the following theorem, where $P$ is a permutation
 matrix which is multiplied to the left side, and $Q$ is sum of elementary matrix and
 certain binary matrix. In which $P$ generalize row permutation on cases of $C$
 matrix,and $Q$ is the transform from $C'$ to C', which is the first n-1 columns
  and sum of certain columns. We aim to find C' such that $C'$ majors $C$,i.e. $C\leq C'$
 
The following theorem is from [].\cite{content}

\begin{thm}\label{pre_thm}
    Let $C=(c_{ij})$, $C'=(c'_{ij})$, $P$ and $Q$ be  $n\times n$ matrices.
Assume that
\begin{enumerate}[label=(\Roman*)]
    \item \label{pre_thm_em1}  $PCQ\leq PC'Q$;
    \item \label{pre_thm_em2} there exist a nonnegative column vector $u=(u_1, u_2, \ldots, u_n)^T$  and a
    scalar $\lambda'\in \mathbb{R}$ such that $\lambda'$ is an eigenvalue of $C'$ with
    associated eigenvector $Qu$;
    \item \label{pre_thm_em3} there exist a nonnegative row vector $v^T=(v_1, v_2, \ldots, v_n)$  and a scalar
    $\lambda\in \mathbb{R}$such that $\lambda$ is an eigenvalue of $C$ with associated  left
    eigenvector $v^TP$; and
    \item \label{pre_thm_em4}$v^TPQu>0.$
\end{enumerate}
    Then $\lambda\leq \lambda'$. Moreover, $\lambda=\lambda'$ if and only if
    \begin{equation}\label{pre0}
        (PC'Q)_{ij}=(PCQ)_{ij}\qquad \hbox{for~}1\leq i, j\leq n \hbox{~with~} v_i\ne 0 \hbox{~and~} u_j\ne 0.
    \end{equation}
\end{thm}
\begin{pof}
    Multiplying the nonnegative vector $u$ in theorem~\ref{pre_thm} assumption
    \ref{pre_thm_em1}, where $Qu$ is eigenvector of $C'$,  to the right of both terms of
    \ref{pre_thm_em1},    
    \begin{equation}\label{pre1}
       PCQu\leq PC'Qu=\lambda'PQu.
    \end{equation}
    Multiplying the nonnegative left eigenvector $v^T$ of $C$ for $\lambda$ in assumption
     \ref{pre_thm_em3} to the left of all terms  in (\ref{pre1}), where $v^TP$ is
    left eigenvector of $C$ for $\lambda$, thus we have
    \begin{equation}\label{pre2}
        \lambda v^TPQu=v^TPCQu\leq v^TPC'Qu=\lambda' v^TPQu.
    \end{equation}
        Now delete the positive term $v^TPQu$ by assumption \ref{pre_thm_em4} to obtain
        $\lambda\leq \lambda'$ and finish the proof of the first part.
        Assume that $\lambda=\lambda'$, so the inequality in (\ref{pre2}) is an equality.
        Especially $(PCQu)_i=(PC'Qu)_i$ for any $i$ with $v_i\not=0.$ Hence,
        $(PCQ)_{ij}=(PC'Q)_{ij}$ for any $i$ with $v_i\not=0$ and any $j$ with
        $u_j\not=0.$ Conversely, (\ref{pre0}) implies $$v^TPCQu=\sum_{i,j} v_i(PCQ)_{ij}u_j=
         \sum_{i,j} v_i(PC'Q)_{ij}u_j=v^TPC'Qu,$$ so $\lambda=\lambda'$ by (\ref{pre2}).


\section{Our Method}
We use [n-1] as notation of the set of elements from one to n-1, which is {1,2,$...$,n}.
Throughout fix $k\in [n-1]$.Let $E_{kn}$ denote the $n\times n$ binary matrix with a unique $1$ appearing in the position $k,n$ of $E_{kn}$. We will apply the previous theorem with $P=I$ and 



\begin{equation}
Q=I+E_{kn}=\begin{pmatrix} \label{Q_1}
1 &  & & &  & 0 \\
 & 1 &  &      &  &  \\
 &  & \ddots & &  & 1 \\
 &  &        & &  &  \\
  &  & & & 1 &  \\
0 &  & & &  & 1 \\
\end{pmatrix}.
\end{equation}






\begin{defn}[$k$-rooted vector]
~A column vector $v'=(v'_1,v'_2,\ldots,v'_n)^T$ is called {\it $k$-rooted}  if $v'_{j} \geq 0$ for $1 \leq  j \leq n$ and $v'_k\geq v'_n.$
\end{defn}
\bigskip

The following Lemma is immediate from the above definition.
\bigskip

\begin{lem}[vector rooted lemma]\label{lem:rt_vec}
If $u=(u_1, u_2, \ldots, u_n)^T$ and $v'=(v'_1, v'_2, \ldots, v'_n):=Qu=(u_1,\ldots, u_{k-1},u_k+u_n, u_{k+1}, \ldots,  u_n)^T$, then
\begin{enumerate}[label=(\Roman*)]
\item \label{lem:rt_vec:en1}$v'$ is $k$-rooted  if and only if  $u$ is nonnegative;
\item $u_k>0$ if and only if $v'_k>v'_n$.
\end{enumerate}
\qed
\end{lem}

\begin{thm}[Our first result]\label{thm_main}
    Let $C=(c_{ij})$, $C'=(c'_{ij})$ be  $n\times n$ matrices.
Assume that
\begin{enumerate}[label=(\Roman*)]
\item \label{thm_main:condition_i} $C[[n]|[n-1]]\leq C'[[n]|[n-1]]$ and $c_{ik}+c_{in}\leq c'_{ik}+c'_{in}$ for all $1\leq i\leq n$;
\item \label{thm_main:condition_ii} there exists a $k$-rooted vector $v'=(v'_1, v'_2, \ldots, v'_n)^T$ and a scalar $\lambda'\in \mathbb{R}$
such that $\lambda'$ is an eigenvalue of $C'$ with associated eigenvector $v'$;
\item there exists a nonnegative vector $v^T=(v_1, v_2, \ldots, v_n)$ and a scalar $\lambda\in \mathbb{R}$ such that $\lambda$ is an eigenvalue of $C$ with associated left eigenvector $v^T$;
\item $v^Tv'>0.$
\end{enumerate}
 Then $\lambda\leq \lambda'$.
Moreover, $\lambda=\lambda'$
if and only if
\begin{enumerate}[label=(\alph*)]
    \item \label{thm_main:equ_cond_a} $c_{ik}+c_{in}=c'_{ik}+c'_{in} \qquad$  for $1\leq i\leq n$ with $v_i\not=0$ and $v'_n\not=0;$
    \item \label{thm_main:equ_cond_b} $c'_{ij}=c_{ij}\qquad $for $1\leq i\leq n,~1\leq j\leq n-1, j \neq k $with $v_i\ne 0 $;
    \item \label{thm_main:equ_cond_c} $c'_{ik}=c_{ik} \qquad $  for $1\leq i \leq n$ and $ v'_{k}>v'_n$ 
\end{enumerate} \qed
\end{thm}
% 



% {}
   
\begin{pof}
    The proof is based on Theorem~\ref{pre_thm} with $P = I$ and $Q = I + E_{kn} \text{ in } (\ref{Q_1} )$. 
    The assumption \ref{pre_thm_em1} $PCQ\leq PC'Q$ of Theorem~\ref{pre_thm} holds by the condition \ref{thm_main:condition_i} of this theorem. 
    Let $u = Q^{-1}v'$. Then u is nonnegative and $C'Qu = \lambda' Qu$ by the condition \ref{thm_main:condition_ii} and
     Lemma~\ref{lem:rt_vec}\ref{lem:rt_vec:en1}. Hence the assumption \ref{pre_thm_em2} of Theorem~\ref{pre_thm} holds. The assumptions \ref{pre_thm_em3} and \ref{pre_thm_em4}
      of Theorem~\ref{pre_thm} clearly hold by conditions~\ref{thm_main:condition_iii},\ref{thm_main:condition_iv} of this theorem since $P = I$ and
       $v'= Qu$  Hence $\lambda \leq \lambda' $ by the necessary condition of Theorem~\ref{pre_thm}. Moreover
        $\lambda = \lambda'$ if and only if \ref{pre0} holds, and this is equivalent to
         conditions \ref{thm_main:equ_cond_a},\ref{thm_main:equ_cond_b},\ref{thm_main:equ_cond_c} of this theorem. 
    \end{pof}   
    
    We are interested in the matrices $C'$ that have $k$-rooted eigenvectors.
    Motivated by the condition (i) of theorem 2.3, we provide the following two definitions. 

    \begin{defn}[(k,n)-sum]
        For an $n \times n$ matrix $C'=(c'_{ij})$, the $(k, n)$-{\it sum} vector of $C'$ is the vector of the sum of the $k$-th and  $n$-th columns of $C'$, where $k\leq n-1$.
    \end{defn}

    Note that the last column of $C'Q$ is the $(k, n)$-{\it sum} vector of $C'$

    \begin{defn}[k-rooted matrix]\label{m_rooted}
        A  matrix $C'=(c'_{ij})$ is called {\it $k$-rooted}  if its  columns and its $(k, n)$-sum vector are all $k$-rooted except the last column of $C'$.
    \end{defn}

    The following theorem shows that a $k$-rooted matrix has a $k$-rooted eigenvector.
    \begin{lem}\label{lma_m_rooted}
        Let $C'=(c'_{ij})$ be an $n\times n$ nonnegative matrix. Then the following (i)-(iii) hold.
            \begin{enumerate}[label=(\Roman*)]
                \item $C'$ is a $k$-rooted matrix, if and only if, $Q^{-1}C'Q$ is nonnegative.
                \item Assume that $C'$ is $k$-rooted and let $u$ be a nonnegative eigenvector of $Q^{-1}C'Q$
                 for $\rho(C')$. Then  $C'$ has a $k$-rooted eigenvector $v'=Qu$ for $\rho(C')$. 
                \item $\rho(C')$ = $\rho(Q^{-1}C'Q)$
            \end{enumerate}
    \end{lem}

% 


    \begin{rem}

    $$Q^{-1}=I-E_{kn}=\begin{pmatrix}
    1 &  & & &  & 0 \\
    & 1 &  &      &  &  \\
    &  & \ddots & &  & -1 \\
    &  &        & &  &  \\
    &  & & & 1 &  \\
    0 &  & & &  & 1 \\
    \end{pmatrix}.$$

    The matrix $C'Q$ is
    $$\begin{pmatrix}
    c'_{11}     & c'_{12} & \cdots     & c'_{1\ n-1} & c'_{1k}+c'_{1n} \\
    \vdots \\
    c'_{k-11}     & c'_{k-1 2}           & \cdots     & c'_{k-1 n-1} & c'_{k-1k}+c'_{k-1n} \\
    c'_{k1} & c'_{k2} &\cdots      & c'_{kn-1} & c'_{kk}+c'_{kn}\\
    c'_{k+11}     & c'_{k+12}           & \cdots     & c'_{k+1\ n-1} & c'_{k+1k}+c'_{k+1n} \\
    \vdots              & \vdots & \ddots              & \vdots & \vdots \\
    c'_{n1}             & c'_{n2} & \cdots             & c'_{n\ n-1} & c'_{nk}+c'_{nn} \\
    \end{pmatrix}.
    $$
    \end{rem}

\begin{comment}


\begin{rmk}
        (i) is immediate from Definition~\ref{m_rooted} and the observation that   
        $$Q^{-1}=I-E_{kn}=\begin{pmatrix}
        1 &  & & &  & 0 \\
         & 1 &  &      &  &  \\
         &  & \ddots & &  & -1 \\
         &  &        & &  &  \\
          &  & & & 1 &  \\
        0 &  & & &  & 1 \\
        \end{pmatrix},$$
\end{rmk}
 \end{comment}   



    \begin{pof}
        and $Q^{-1}C'Q$ is
        $$\begin{pmatrix}
        c'_{11}     & c'_{12} & \cdots     & c'_{1\ n-1} & c'_{1k}+c'_{1n} \\
        \vdots \\
        c'_{k-11}     & c'_{k-1 2}           & \cdots     & c'_{k-1 n-1} & c'_{k-1k}+c'_{k-1n} \\
        c'_{k1}-c'_{n1} & c'_{k2}-c'_{n2} &\cdots      &c'_{kn-1}-c'_{nk-1}& c'_{kk}+c'_{kn}-c'_{nk}-x'_{nn}\\
        c'_{k+11}     & c'_{k+12}           & \cdots     & c'_{k+1\ n-1} & c'_{k+1k}+c'_{k+1n} \\
        \vdots              & \vdots & \ddots              & \vdots & \vdots \\
        c'_{n1}             & c'_{n2} & \cdots             & c'_{n\ n-1} & c'_{nk}+c'_{nn} \\
        \end{pmatrix}.$$
    \end{pof}

    \begin{pof}
        (ii)
            By Lemma~\ref{lem:rt_vec} $v'=Qu$ is $k$-rooted.  
            Since $Q^{-1}C'Qu=\rho(C')u$ by the assumption, we have
            $Q^{-1} C' Q u  = Q^{-1} \rho(C') Qu  =\rho(C')u$  \\
            $C'Qu=\rho(C')Qu$.

        (iii)
        Since $C'$ and $Q^{-1}C'Q$ have the same set of eigenvalues, clearly $\rho(C')$ = $\rho(Q^{-1}C'Q)$.

    \end{pof}


\begin{lem}\label{l_diag}   %end of label
If a square matrix $C'$ has a rooted eigenvector for $\lambda'$, then $C'+dI$ also has
the same rooted eigenvector for $\lambda'+d,$ where $d$ is a constant and $I$ is the identity matrix with the same size of $C'$.
\end{lem}

\begin{thm}
    Let $C$ be an $n\times n$ nonnegative matrix. For $1\leq i \leq n$ and $1\leq j\leq n-1$, choose $c'_{ij}$
    such that $c'_{ij}\geq c_{ij}$ and $c'_{kj}\geq c'_{nj}>0$, and choose $r'_i$ such that $r'_i\geq c_{ik}+c_{in}$, and
    $r'_k \geq r'_n$. Moreover choose $c'_{in}:=r'_i-c'_{ik}$. Then $\rho(C)\leq \rho(C')$, when $C'=(c'_{ij})$.
\end{thm}

\begin{pof}
    These assumptions are necessary that $PCQ \leq PC'Q$, and C' is $k$-rooted,based on
    (\ref{lma_m_rooted});\\
    calculation:  \\
       \\
    by \ref{l_diag}, For certain d, if $C'+dI$ is $k$-rooted, then it has
     a $k$-rooted eigenvector with its spectral radius $\lambda + d$. C' would share the same
      eigenvector with $C'+dI$ and has eigenvalue $\lambda$. So $C'+dI$ and $C+dI$ meet the
       conditions of \ref{thm_main}, and we can show that $\rho(C' + dI) \geq \rho(C +dI)$ and
        then $\rho(C') \geq \rho(C)$  \qed
\end{pof}



\subsection{Example}
    For the following $4\times 4$ matrix
    $$C=\begin{pmatrix}
    0 & 0 & 1 & 1\\
    1 & 0 & 0 & 1\\
    1 & 1 & 0 & 1\\
    1 & 1 & 1 & 0
    \end{pmatrix},$$
    we choose
    $$C'=\begin{pmatrix}
    0 & 0 & 1 & 1\\
    1 & 0 & 1 &  0\\
    1 & 1 & 0 & 1\\
    1 & 1 & 1 & 0
    \end{pmatrix}.$$
    Check the conditions $C[[n]|[n-1]]  \leq C'[[n]|[n-1]] $ \ref{thm_main},
    Then
    $\rho(C)\leq \rho(C')$ by previous theorem.



\subsection{Counterexample}
    For the following two $4\times 4$ matrices
    $$C=\begin{pmatrix}
    0 & 0 & 1 & 1\\
    1 & 0 & 0 & 1\\
    1 & 1 & 0 & 0\\
    1 & 1 & 1 & 0
    \end{pmatrix},\quad C'=\begin{pmatrix}
    0 & 0 & 1 & 1\\
    1 & 0 & 1 &  0\\
    1 & 1 & 0 & 0\\
    1 & 1 & 1 & 0
    \end{pmatrix},$$ 
    specify $n$=4, $k$=3 in $Q = I +E_{kn} = I + E_{34}$  
    $$CQ=\begin{pmatrix}
    0 & 0 & 1 & 2\\
    1 & 0 & 0 & 1\\
    1 & 1 & 0 & 0\\
    1 & 1 & 1 & 1
    \end{pmatrix},\quad C'Q=\begin{pmatrix}
    0 & 0 & 1 & 2\\
    1 & 0 & 1 & 1\\
    1 & 1 & 0 & 0\\
    1 & 1 & 1 & 1
    \end{pmatrix},$$

    we have $CQ\leq C'Q$, but 
    $\rho(C)=2.234\not\leq 2.148= \rho(C')$. 
    This is because $c'_{33}+c'_{34}\not\geq c'_{43}+c'_{44}$. 


\section{References}
\begin{thebibliography}{20}
\normalsize
\addcontentsline{toc}{chapter}{Bibliography}

\bibitem{chang}
Yen-Jen Cheng, {\it  A matrix realization of spectral bounds
of the spectral radius of a nonnegative matrix}, Ph.D. Thesis, NCTU, 2018.

\bibitem{prn_fros1}
A. E. Brouwer, W. H. Haemers, {\it Spectra of graphs}, Springer, 2012

\bibitem{prn_fros2}
R. A. Horn , C. R. Johnson, {\it Matrix analysis}, Cambrigde University Press, 1985.

\end{thebibliography}



\end{document}
