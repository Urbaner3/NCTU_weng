\documentclass[12pt, a4paper]{article}



\setlength{\textwidth}{160mm} \setlength{\textheight}{220mm}
\topmargin=-0.0cm \oddsidemargin=-0.1cm \setcounter{page}{1}


\usepackage{fontspec}   %加這個就可以設定字體
\usepackage{xeCJK}       %讓中英文字體分開設置
%\usepackage{times}
\usepackage{amssymb}
\usepackage{amsmath}
\usepackage{amsthm}
\usepackage{ulem}
\usepackage{amsfonts}
\usepackage{mathrsfs}
\usepackage{tabularx,array}
\usepackage{pgf,tikz}
 \usepackage{blkarray} %line 80-92 Q formula matrix index
%\usepackage{colortbl}
\usepackage{enumitem}% enumerate labels   roman
\usepackage{graphicx}

\graphicspath{{images/}}
\usetikzlibrary{arrows}
\setCJKmainfont{標楷體} %設定中文為系統上的字型,而英文不去更動,使用原TeX字型
\XeTeXlinebreaklocale "zh"             %這兩行一定要加,中文才能自動換行
\XeTeXlinebreakskip = 0pt plus1pt     %這兩行一定要加,中文才能自動換行
\title{Combinatorial Identities from Lagrange's Interpolation Polynomial}
\author{Student: Yen-Jung Huang  ~~~~~~~~~~~~~~~~~~~~~~~~~~Advisor: Chih-Wen Weng}
\date{} %不要日期

\def\UrlFont{\rm}


\theoremstyle{plain}
\newtheorem{thm}{Theorem}[section]
\newtheorem{cor}[thm]{Corollary}
\newtheorem{lem}[thm]{Lemma}
\newtheorem{prop}[thm]{Proposition}
\newtheorem{remark}[thm]{Remark}
\newtheorem{eg}[thm]{Example}
\newtheorem{conj}[thm]{Conjecture}


\theoremstyle{definition}
\newtheorem{defn}[thm]{Definition}
\newtheorem{ex}[thm]{Exercise}
\newtheorem{prob}[thm]{Problem}
\newtheorem{exam}[thm]{Example}
\newtheorem{nota}[thm]{Notation}
\newtheorem{rem}[thm]{Remark}
\newtheorem{ques}[thm]{Question}
% \newtheorem{pof}[thm]{Proof.}

\usepackage{comment}

%\renewcommand {\refname} {Bibliography}



\begin{document}

%封面

\thispagestyle{empty}
\begin{center}
{ \Huge 國~~~~立~~~~交~~~~通~~~~大~~~~學}~\\~\\

\bigskip

{ \Huge 應~用~數~學~系}~\\~\\

\bigskip

{ \Huge 碩~~士~~論~~文}~\\~\\

\bigskip \bigskip\bigskip\bigskip\bigskip\bigskip

{ \Huge 有向圖譜半徑之簡易估算方法}~\\~\\

\bigskip

{ \Huge A simple method on estimating spectral radius for some directed graphs}~\\~\\
~\\~\\~\\~\\~\\
\bigskip \bigskip\bigskip\bigskip\bigskip\bigskip
\bigskip \bigskip\bigskip\bigskip\bigskip\bigskip
\bigskip\bigskip\bigskip

{ \Large
\begin{tabular}{rcl}
研究生&:&陳科翰\\
指導教授&:&翁志文~教授
\end{tabular} }

\bigskip\bigskip
{ \Large 中~~華~~民~~國~~一~~百~~ㄧ~~十~~年~~一~~月 }
\large
\end{center}
\pagebreak




%%%%%%%%%%%%%%%%%%%%%%%%%%%%%%%%%%%%%%%%%%%%%%%%%%%%%%%%%%%%%%%%%%%%%%%%%
\renewcommand{\baselinestretch}{2} %行距
\thispagestyle{empty}
\begin{center}
{
\Large
有向圖譜半徑之簡易估算方法\\
A simple method on estimating spectral radius for some directed graphs\\~\\
\begin{tabular}{lccr}
Student: Ko-Han Chen  &&~~~& Advisor: Chih-Wen Weng\\
研究生:陳科翰  &&~~~& 指導教授:翁志文~教授
\end{tabular}
}~\\

\bigskip

\renewcommand{\baselinestretch}{1} %行距

{ \LARGE 國~~~~立~~~~交~~~~通~~~~大~~~~學}\\~\\
{ \LARGE 應~用~數~學~系}\\~\\
{ \LARGE 碩~~士~~論~~文}\\~\\~\\~\\~\\
\renewcommand{\baselinestretch}{1} %行距
{ \large A Thesis

Submitted to Department of Applied Mathematics

College of Science

National Chiao Tung University

in Partial Fulfillment of Requirements

for the Degree of Master

in Applied Mathematics
\bigskip \medskip

January 2021

Hsinchu, Taiwan, Republic of China \bigskip \medskip

 中~~華~~民~~國~~一~~百~~ㄧ~~十~~年~~一~~月 }
\end{center}
\pagebreak

\label{abstract}

%中文摘要
\pagenumbering{roman}
\begin{center}
{  \LARGE
有向圖譜半徑之簡易估算方法
\bigskip\bigskip\bigskip

研究生:陳科翰  ~~~~~~~~~~ 指導教授:翁志文~教授 \\
國立交通大學  \\
\bigskip
應用數學系
\bigskip\bigskip\bigskip\bigskip
} \\~\\~\\~\\
\addcontentsline{toc}{section}{Abstract (in Chinese)}
{\large 摘~要}
\end{center}
 \bigskip

 矩陣的譜半徑為其特徵值絕對值的最大值,而一有向圖的譜半徑則定義為其鄰接矩陣之譜半徑。本論文給出一個尋找方陣譜半徑上界的方法,將此方法應用於有向圖的鄰接矩陣,我們可以簡易估算出此圖的譜半徑。
 \\
\bigskip

\noindent 關鍵詞:譜半徑、鄰接矩陣
\pagebreak



%英文摘要
\begin{center}{\LARGE
A simple method on estimating spectral radius for some directed graphs
\bigskip\bigskip\bigskip}

{ \large
Student: Ko-Han Chen  ~~~~~ Advisor: Chih-Wen Weng \\
\Large

Department ~of~ Applied ~Mathematics
\bigskip

National~ Chiao ~Tung~ University
\bigskip\bigskip\bigskip\bigskip}\\
{\large Abstract}
\end{center}

%\begin{abstract}
\addcontentsline{toc}{section}{Abstract (in English)}
The spectral radius of a square matrix is the largest magnitude of its eigenvalues. And the spectral radius of a directed graph is defined as the spectral radius of the corresponding adjacency matrix. In this paper, we give an approach to obtain an upper bound for the spectral radius of a square matrix. By applying this method on the adjacency matrix of a directed graph, we can estimate the spectral radius of this directed graph simply.

\bigskip


\noindent {\bf Keywords}: spectral radius, adjacency matrix
%\end{abstract}
\pagebreak


\renewcommand{\baselinestretch}{1.2}
% 目錄
\large
\tableofcontents


\clearpage
\pagenumbering{arabic}
\linespread{1.5}

\setcounter{page}{1}
%%%%%%%%%%%%%%%%%%%%%%
\section{Introduction}
%%%%%%%%%%%%%%%%%%%%%%

Let $\mathbb{R}$ and $\mathbb{C}$ denote the field of real numbers and complex numbers, respectively. Let $C$ be an $n\times n$ real square matrix. If there is a nonzero column vector $u\in\mathbb{R}^n$ such that $Cu=\lambda u$ for some scalar $\lambda\in\mathbb{C}$, then the scalar $\lambda$ is called the eigenvalue of $C$ with corresponding eigenvector $u$. And the spectral radius of a matrix $C$ is the largest magnitude (or complex modulus) of its eigenvalues, denoted by $\rho(C)$. We are interested in the spectral radius of the following matrix associated with a simple directed graph.

\begin{defn}
    Given an directed graph $G$, the $\textit{adjacency matrix}$ of $G$ is the square
    matrix $A = (a_{ij})$ indexed by vertices of $G$, and
     \[a_{ij} =\begin{cases}
        1, \text{if $i$ is adjacent to $j$}, \\
        0, \text{otherwise.}
            \end{cases}
     \]
\end{defn}

Given a directed graph $G$, the spectral radius of $G$ is the spectral radius of the adjacency matrix of $G$, denoted by $\rho(G)$. Note that the spectral radius $\rho(G)$ is independent of the ordering of the vertex set of $G$.

The spectral radius is an important indicator to specify the relation of connected vertices in a graph, so it is meaningful to find a simple method to estimate the spectral radius. A simple and excellent executable method to estimate the spectral radius has some features, first, the bios is minimized, and second, there must be a way to prove it sensible. Enumerate these factors and prove it correctly would make this method reliable.

In \cite{chang}, Cheng and Weng give many bounds of the spectral radius of a nonnegative square matrix. And based on their theory and Perron-Frobenius theorem, we give another approach to obtain an upper bound of the spectral radius and apply it on the adjacency matrix of a directed graph.

%%%%%%%%%%%%%%%%%%%%%%%
\section{Preliminaries}
%%%%%%%%%%%%%%%%%%%%%%%

The following is Perron–Frobenius theorem, which provides a feature of
 nonnegative eigenvector to nonnegative matrices.

\begin{thm} \cite{prn_fros2} \label{thm:Perron_Frobenius}
    If $C$ is a nonnegative square matrix, then the spectral radius $\rho(C)$ is an
    eigenvalue of $C$ with a corresponding nonnegative right eigenvector and a
    corresponding nonnegative left eigenvector.
\end{thm}

We introduce a notation of submatrix, which is taken from some columns and some rows of a matrix.

\begin{defn}
    For a matrix $C=(c_{ij})$ and subsets $\alpha$, $\beta$ of row indices and column
    indices of $C$, respectively, we use $C[\alpha|\beta]$ to denote the
    submatrix of $C$ with size $ |\alpha| \times |\beta| $ that has entries $c_{ij}$ for $i\in \alpha$
    and $j\in\beta$.
\end{defn}

Our theory is based on the following theorem, which is from \cite{chang}, and we introduce two square matrices $P$ and $Q$ in the following theorem first. The matrix $P$ is a permutation matrix which is multiplied to the left side, and the matrix $Q$ is the sum of an elementary matrix and a certain binary matrix. In which $P$ generalize row permutation on cases of $C$ matrix, and $Q$ is the transform from $C'$ to $C'$, which is the first $n-1$ columns and sum of certain columns. We aim to find $C'$ such that $C'$ majors $C$, i.e. $C\leq C'$.

\begin{thm}\label{pre_thm}
    Let $C=(c_{ij})$, $C'=(c'_{ij})$, $P$ and $Q$ be $n\times n$ matrices.
Assume that
\begin{enumerate}[label=(\roman*)]
    \item \label{pre_thm_em1}  $PCQ\leq PC'Q$;
    \item \label{pre_thm_em2} there exist a nonnegative column vector $u=(u_1, u_2, \ldots, u_n)^T$  and a
    scalar $\lambda'\in \mathbb{R}$ such that $\lambda'$ is an eigenvalue of $C'$ with
    associated eigenvector $Qu$;
    \item \label{pre_thm_em3}  there exist a nonnegative row vector $v^T=(v_1, v_2, \ldots, v_n)$  and a scalar
    $\lambda\in \mathbb{R}$such that $\lambda$ is an eigenvalue of $C$ with associated  left
    eigenvector $v^TP$; and
    \item \label{pre_thm_em4} $v^TPQu>0.$
\end{enumerate}
    Then $\lambda\leq \lambda'$. Moreover, $\lambda=\lambda'$ if and only if
    \begin{equation}\label{pre0}
        (PC'Q)_{ij}=(PCQ)_{ij}\qquad \hbox{for~}1\leq i, j\leq n \hbox{~with~} v_i\ne 0 \hbox{~and~} u_j\ne 0.
    \end{equation}
\end{thm}

\begin{proof}
    Multiplying the nonnegative vector $u$ in assumption~\ref{pre_thm_em2}, where $Qu$ is eigenvector of $C'$, to the right of both terms of
    ~\ref{pre_thm_em1},
    \begin{equation}\label{pre1}
       PCQu\leq PC'Qu=\lambda'PQu.
    \end{equation}
    Multiplying the nonnegative left eigenvector $v^T$ of $C$ for $\lambda$ in assumption
     ~\ref{pre_thm_em3} to the left of all terms  in (\ref{pre1}), where $v^TP$ is
    left eigenvector of $C$ for $\lambda$, thus we have
    \begin{equation}\label{pre2}
        \lambda v^TPQu=v^TPCQu\leq v^TPC'Qu=\lambda' v^TPQu.
    \end{equation}
        Now delete the positive term $v^TPQu$ by assumption \ref{pre_thm_em4} to obtain
        $\lambda\leq \lambda'$ and finish the proof of the first part.
        Assume that $\lambda=\lambda'$, so the inequality in (\ref{pre2}) is an equality.
        Especially $(PCQu)_i=(PC'Qu)_i$ for any $i$ with $v_i\not=0.$ Hence,
        $(PCQ)_{ij}=(PC'Q)_{ij}$ for any $i$ with $v_i\not=0$ and any $j$ with
        $u_j\not=0.$ Conversely, (\ref{pre0}) implies $$v^TPCQu=\sum_{i,j} v_i(PCQ)_{ij}u_j=
         \sum_{i,j} v_i(PC'Q)_{ij}u_j=v^TPC'Qu,$$ so $\lambda=\lambda'$ by (\ref{pre2}).
\end{proof}

%%%%%%%%%%%%%%%%%%%%
\section{Our Method}
%%%%%%%%%%%%%%%%%%%%

We use $[n-1]$ as notation of the set of elements from one to $n-1$, which is $\{1,2,...,n-1\}$.
Throughout fix $k\in [n-1]$, let $E_{kn}$ denote the $n\times n$ binary matrix with a unique $1$
appearing in the position $k,n$ of $E_{kn}$. Now we apply the previous Theorem \ref{pre_thm} with $P=I$ and


\begin{equation} \label{Q_1}
Q=I+E_{kn}=\begin{pmatrix}
1 &   &         &   & 0 \\
  & 1 &         &   &   \\
  &   & \ddots  &   & 1 \\
  &   &         & 1 &  \\
0 &   &         &     & 1
\end{pmatrix},
\end{equation}
so the matrix $PC'Q$ in assumption~\ref{pre_thm_em1} of Theorem~\ref{pre_thm} is

\begin{equation}\label{PC'Q}
  PC'Q=\begin{pmatrix}
         c'_{11} & c'_{12} & \cdots &  c'_{1k}+c'_{1n} \\
         c'_{21} & c'_{22} & \cdots &  c'_{2k}+c'_{2n} \\
         \vdots & \vdots & \ddots &  \vdots \\
         c'_{n1} & c'_{n2} & \cdots &  c'_{nk}+c'_{nn}
       \end{pmatrix},
\end{equation}
where $c'_{ij}$ denote the $(i,j)$-entry of $C$.






\begin{defn}%[$K$-rooted vector]
A column vector $v'=(v'_1,v'_2,\ldots,v'_n)^T$ is called {\it $K$-rooted}  if $v'_{j} \geq 0$ for $1 \leq  j \leq n$ and $v'_k\geq v'_n.$
\end{defn}

The following Lemma is immediate from the above definition.%[vector rooted lemma]

\begin{lem}\label{lem:rt_vec}
If $u=(u_1, u_2, \ldots, u_n)^T$ and $v'=(v'_1, v'_2, \ldots, v'_n):=Qu=(u_1,\ldots, u_{k-1},u_k+u_n, u_{k+1}, \ldots,  u_n)^T$, then
\begin{enumerate}[label=(\roman*)]
\item \label{lem:rt_vec:en1}$v'$ is $k$-rooted  if and only if $u$ is nonnegative;
\item $u_k>0$ if and only if $v'_k>v'_n$.
\end{enumerate}
% \qed
\end{lem}

Below is our first result, in which the first condition implies the first $n-1$ columns of $C$ major to
 the columns of $C'$, and the $(k,n)$-sum column of $C$ is also major to $C'$. The second and the third condition
 suggest that $C$ and $C'$ have nonnegative eigenvectors which are $K$-rooted. And the forth condition is simpler
  but with the same meaning with Theorem~\ref{pre_thm}
\begin{thm}\label{thm_main}
    Let $C=(c_{ij})$, $C'=(c'_{ij})$ be  $n\times n$ matrices.
Assume that
\begin{enumerate}[label=(\roman*)]
\item \label{thm_main:condition_i} $C[[n]|[n-1]]\leq C'[[n]|[n-1]]$ and $c_{ik}+c_{in}\leq c'_{ik}+c'_{in}$ for all $1\leq i\leq n$;
\item \label{thm_main:condition_ii} there exists a $K$-rooted vector $v'=(v'_1, v'_2, \ldots, v'_n)^T$ and a scalar $\lambda'\in \mathbb{R}$
such that $\lambda'$ is an eigenvalue of $C'$ with associated eigenvector $v'$;
\item \label{thm_main:condition_iii}there exists a nonnegative vector $v^T=(v_1, v_2, \ldots, v_n)$ and a scalar $\lambda\in \mathbb{R}$ such that $\lambda$ is an eigenvalue of $C$ with associated left eigenvector $v^T$;
\item \label{thm_main:condition_iv}$v^Tv'>0.$
\end{enumerate}
 Then $\lambda\leq \lambda'$.
Moreover, $\lambda=\lambda'$
if and only if
\begin{enumerate}[label=(\alph*)]
    \item \label{thm_main:equ_cond_a} $c_{ik}+c_{in}=c'_{ik}+c'_{in} \qquad$  for $1\leq i\leq n$ with $v_i\not=0$ and $v'_n\not=0;$
    \item \label{thm_main:equ_cond_b} $c'_{ij}=c_{ij}\qquad $for $1\leq i\leq n,~1\leq j\leq n-1, j \neq k $with $v_i\ne 0 $;
    \item \label{thm_main:equ_cond_c} $c'_{ik}=c_{ik} \qquad $  for $1\leq i \leq n$ and $ v'_{k}>v'_n$
\end{enumerate} %\qed
\end{thm}
%


\begin{proof}
    The proof is based on Theorem~\ref{pre_thm} with $P = I$ and $Q = I + E_{kn}$ in (\ref{Q_1}).
    The assumption \ref{pre_thm_em1} $PCQ\leq PC'Q$ of Theorem~\ref{pre_thm} holds by the condition \ref{thm_main:condition_i} of this theorem.
    Let $u = Q^{-1}v'$. Then u is nonnegative and $C'Qu = \lambda' Qu$ by the condition \ref{thm_main:condition_ii} and \ref{lem:rt_vec:en1} in
     Lemma~\ref{lem:rt_vec}. Hence the assumption \ref{pre_thm_em2} of Theorem~\ref{pre_thm} holds. The assumptions \ref{pre_thm_em3} and \ref{pre_thm_em4}
      of Theorem~\ref{pre_thm} clearly hold by conditions~\ref{thm_main:condition_iii}, \ref{thm_main:condition_iv} of this theorem since $P = I$ and
       $v'= Qu$. Hence $\lambda \leq \lambda' $ by the necessary condition of Theorem~\ref{pre_thm}. Moreover,
        $\lambda = \lambda'$ if and only if (\ref{pre0}) holds, and this is equivalent to
         conditions \ref{thm_main:equ_cond_a}, \ref{thm_main:equ_cond_b} and \ref{thm_main:equ_cond_c} of this theorem.
\end{proof}

We are interested in the matrices $C'$ that have $K$-rooted eigenvectors.
Motivated by the condition (i) of Theorem 2.3, we provide the following two definitions.
This is the definition of $(k,n)$-sum.
\begin{defn}
    For an $n \times n$ matrix $C'=(c'_{ij})$, the $(k, n)$-sum vector of $C'$ is the vector of the sum of the $k$-th and  $n$-th columns of $C'$, where $k\leq n-1$.
\end{defn}

Note that the last column of $C'Q$ is the $(k, n)$-sum vector of $C'$.
Below is the definition of $K$-rooted matrix.
\begin{defn}\label{m_rooted}
    A  matrix $C'=(c'_{ij})$ is called {\it $K$-rooted} if its first $n-1$ columns and its $(k,n)$-sum vector are all $K$-rooted.
\end{defn}


\begin{rem}

    $$Q^{-1}=I-E_{kn}=\begin{pmatrix}
    1 &  & &   & 0 \\
    & 1 &        &  &  \\
    &  & \ddots  &  & -1 \\
    &  & &  1 &  \\
    0 &  &  &  & 1
    \end{pmatrix}.$$

\end{rem}

The following lemma shows that a $K$-rooted matrix has a $K$-rooted eigenvector.
\begin{lem}\label{lma_m_rooted}
    Let $C'=(c'_{ij})$ be an $n\times n$ nonnegative matrix. Then the following (i)-(iii) hold.
        \begin{enumerate}[label=(\roman*)]
            \item \label{lma_m_rooted_cond1} $C'$ is a $K$-rooted matrix if and only if $Q^{-1}C'Q$ is nonnegative,
            \item \label{lma_m_rooted_cond2}Assume that $C'$ is $K$-rooted and let $u$ be the nonnegative eigenvector of $Q^{-1}C'Q$
                corresponding to $\rho(C')$. Then  $C'$ has a $K$-rooted eigenvector $v'=Qu$ for $\rho(C')$,
            \item \label{lma_m_rooted_cond3}$\rho(C')$ = $\rho(Q^{-1}C'Q)$.
        \end{enumerate}
\end{lem}


\begin{comment}
\begin{rem}
        (i) is immediate from Definition~\ref{m_rooted} and the observation that
        $$Q^{-1}=I-E_{kn}=\begin{pmatrix}
        1 &  & & &  & 0 \\
         & 1 &  &      &  &  \\
         &  & \ddots & &  & -1 \\
         &  &        & &  &  \\
          &  & & & 1 &  \\
        0 &  & & &  & 1 \\
        \end{pmatrix},$$
\end{rem}
\end{comment}
\begin{proof}
\begin{enumerate}
  \item[(i)] The matrix $Q^{-1}C'Q$ is
        $$\begin{pmatrix}
            c'_{11}     & c'_{12} & \cdots     & c'_{1\ n-1} & c'_{1k}+c'_{1n} \\
            \vdots      & \vdots  & \vdots     & \vdots      & \vdots\\
            c'_{(k-1) 1}   & c'_{(k-1)  2}           & \cdots     & c'_{(k-1) (n-1)} & c'_{(k-1) k}+c'_{(k-1) n} \\
            c'_{k1}-c'_{n1} & c'_{k2}-c'_{n2} &\cdots      &c'_{k (n-1)}-c'_{n (k-1)}& c'_{kk}+c'_{kn}-c'_{nk}-x'_{nn}\\
            c'_{(k+1) 1}   & c'_{(k+1) 2}           & \cdots     & c'_{(k+1) (n-1)} & c'_{(k+1) k}+c'_{(k+1)
             n} \\
            \vdots              & \vdots & \ddots              & \vdots & \vdots \\
            c'_{n1}     & c'_{n2} & \cdots             & c'_{n (n-1)} & c'_{nk}+c'_{nn} \\
        \end{pmatrix},$$

        which is nonnegative if and only if $C'$ is $K$-rooted by the definition of nonnegative matrix and $K$-rooted matrix.
  \item[(ii)] By Lemma~\ref{lem:rt_vec}, $v'=Qu$ is $K$-rooted. Since $Q^{-1}C'Qu=\rho(C')u$ by the assumption, we have
      $$Q^{-1} C' Q u  = Q^{-1} \rho(C') Qu  =\rho(C')u,$$
      which implies
        $$C'Qu=\rho(C')Qu.$$
  \item[(iii)] Since $C'$ and $Q^{-1}C'Q$ have the same set of eigenvalues, clearly
  $$\rho(C') = \rho(Q^{-1}C'Q).$$
\end{enumerate}
\end{proof}


\begin{lem}\label{l_diag}   %end of label
If a square matrix $C'$ has a $K$-rooted eigenvector corresponding to $\lambda'$, then $C'+dI$ also has
the same $K$-rooted eigenvector corresponding to $\lambda'+d,$ where $d$ is a constant and $I$ is the identity matrix with the same size of $C'$.
\end{lem}

\begin{thm}\label{thm:conclusion}
    Let $C$ be an $n\times n$ nonnegative matrix. For $1\leq i \leq n$ and $1\leq j\leq n-1$, choose $c'_{ij}$
    such that $c'_{ij}\geq c_{ij}$ and $c'_{kj}\geq c'_{nj}>0$, and choose $r'_i$ such that $r'_i\geq c_{ik}+c_{in}$, and
    $r'_k \geq r'_n$. Moreover choose $c'_{in}:=r'_i-c'_{ik}$. Then $\rho(C)\leq \rho(C')$, when $C'=(c'_{ij})$.
\end{thm}

\begin{proof}
    These assumptions are necessary that $PCQ \leq PC'Q$, where $P=I$, $Q=I+E_{kn}$ and $C'$ is $K$-rooted. Based on Lemma~\ref{lma_m_rooted},
    $$Q^{-1}C'Q=
        \begin{pmatrix}
        c'_{11}     & c'_{12} & \cdots     & c'_{1 (n-1)} & c'_{1k}+c'_{1n} \\
        \vdots \\
        c'_{(k-1)1}     & c'_{(k-1) 2}           & \cdots     & c'_{(k-1) (n-1)} & c'_{(k-1)k}+c'_{(k-1)n} \\
        c'_{k1}-c'_{n1} & c'_{k2}-c'_{n2} &\cdots      &c'_{k(n-1)}-c'_{n(k-1)}& c'_{kk}+c'_{kn}-c'_{nk}-c'_{nn}\\
        c'_{(k+1)1}     & c'_{(k+1)2}           & \cdots     & c'_{(k+1) (n-1)} & c'_{(k+1)k}+c'_{(k+1)n} \\
        \vdots              & \vdots & \ddots              & \vdots & \vdots \\
        c'_{n1}             & c'_{n2} & \cdots             & c'_{n (n-1)} & c'_{nk}+c'_{nn}
    \end{pmatrix}.$$
    Then $Q^{-1}C'Q$ is nonnegative when $c'_{kj}\geq c'_{nj}>0$ for $1\leq j\leq n-1$ and $c'_{ij}\geq c_{ij}$, where $C$ is nonnegative, and the last column $c'_{in}+c'_{ik}=r'_i \geq c_{in}+c_{ik}$ by assumption $r'_i\geq c_{ik}+c_{in}$.

    For $1\leq i \leq n$ and $1\leq j\leq n-1$, choose $c'_{ij}$ such that $c'_{ij}\geq c_{ij}$, which implies $C[[n]|[n-1]]\leq C'[[n]|[n-1]]$. And under the same condition for $i$ and $j$, choose $r'_i$ such that $r'_i\geq c_{ik}+c_{in}$, which implies $c_{ik}+c_{in}\leq c'_{ik}+c'_{in} = r'_i$; $C'$ is $K$-rooted matrix, then by conditions \ref{lma_m_rooted_cond2} and \ref{lma_m_rooted_cond3} in Lemma~\ref{lma_m_rooted}, there exists a $K$-rooted vector $v'=(v'_1, v'_2, \ldots, v'_n)^T$ and a scalar $\lambda'\in \mathbb{R}$
    such that $\lambda'$ is an eigenvalue of $C'$ with associated eigenvector $v'$.
   Since $C$ is nonnegative, by Theorem~\ref{thm:Perron_Frobenius}, we claim there exists $v^T = w^{T}P$, such that $v^T$ is nonnegative left eigenvector of $C$, and a scalar $\lambda\in \mathbb{R}$ such that $\lambda$ is an eigenvalue of $C$ with associated left eigenvector $v^T$.
    Due to $v'$ and $v^T$ are nonnegative, $v^Tv'>0$, unless they are orthogonal, i.e, $v^Tv'=0$.

    Here we can summarize the facts we know so far as the following:
    \begin{enumerate}[label=(\roman*)]
        \item  $C[[n]|[n-1]]\leq C'[[n]|[n-1]]$ and $c_{ik}+c_{in}\leq c'_{ik}+c'_{in}$ for all $1\leq i\leq n$;
        \item  there exists a $K$-rooted vector $v'=(v'_1, v'_2, \ldots, v'_n)^T$ and a scalar $\lambda'\in \mathbb{R}$
        such that $\lambda'$ is an eigenvalue of $C'$ with associated eigenvector $v'$;
        \item there exists a nonnegative vector $v^T=(v_1, v_2, \ldots, v_n)$ and a scalar $\lambda\in \mathbb{R}$ such that $\lambda$ is an eigenvalue of $C$ with associated left eigenvector $v^T$;
        \item $v^Tv'>0,$
    \end{enumerate}
    which come from Theorem~\ref{thm_main}.

    By Lemma~\ref{l_diag}, for certain $d$, if $C'+dI$ is $K$-rooted, then it has
    a $K$-rooted eigenvector with its spectral radius $\lambda + d$. $C'$ would share the same
    eigenvector with $C'+dI$ and has eigenvalue $\lambda$. So $C'+dI$ and $C+dI$ meet the
    conditions of Theorem~\ref{thm_main}, and we can show that $\rho(C' + dI) \geq \rho(C +dI)$ and
    then $\rho(C') \geq \rho(C)$  %\qed
\end{proof}


\subsection{Example}
    For the following $4\times 4$ matrix
    $$C=\begin{pmatrix}
    0 & 0 & 1 & 1\\
    1 & 0 & 0 & 1\\
    1 & 1 & 0 & 1\\
    1 & 1 & 1 & 0
    \end{pmatrix},$$
    and corresponding directed graph, \cite[sage]{sage}
    \begin{center}
    \includegraphics{graph_C.PNG}
    \end{center}
    we choose the matrix
    $$C'=\begin{pmatrix}
    0 & 0 & 1 & 1\\
    1 & 0 & 1 & 0\\
    1 & 1 & 0 & 1\\
    1 & 1 & 1 & 0
    \end{pmatrix},$$
    which satisfies the assumptions in Theorem~\ref{thm:conclusion}. Hence $\rho(C)\leq\rho(C')$. Actually, both of the values of $\rho(C)$ and $\rho(C')$ are close to 2.511547, and we have $\rho(C')-\rho(C)>0$ by calculating on computer.
    Moreover, the corresponding directed graph of matrix $C'$ is, \cite[sage]{sage}
    \begin{center}
    \includegraphics{graph_Cprime.PNG}
    \end{center}



\subsection{Counterexample}
    For the following two $4\times 4$ matrices
    $$C=\begin{pmatrix}
    0 & 0 & 1 & 1\\
    1 & 0 & 0 & 1\\
    1 & 1 & 0 & 0\\
    1 & 1 & 1 & 0
    \end{pmatrix},\quad C'=\begin{pmatrix}
    0 & 0 & 1 & 1\\
    1 & 0 & 1 &  0\\
    1 & 1 & 0 & 0\\
    1 & 1 & 1 & 0
    \end{pmatrix},$$
    specify $n$=4, $k$=3 in $Q = I +E_{kn} = I + E_{34}$, then
    $$CQ=\begin{pmatrix}
    0 & 0 & 1 & 2\\
    1 & 0 & 0 & 1\\
    1 & 1 & 0 & 0\\
    1 & 1 & 1 & 1
    \end{pmatrix},\quad C'Q=\begin{pmatrix}
    0 & 0 & 1 & 2\\
    1 & 0 & 1 & 1\\
    1 & 1 & 0 & 0\\
    1 & 1 & 1 & 1
    \end{pmatrix},$$

    and we have $CQ\leq C'Q$, but
    $\rho(C)=2.234\not\leq 2.148= \rho(C')$.
    This is because $c'_{33}+c'_{34}\not\geq c'_{43}+c'_{44}$.




%\section*{References}
\begin{thebibliography}{20}
\normalsize
\addcontentsline{toc}{section}{Bibliography}

\bibitem{chang}
Yen-Jen Cheng, {\it  A matrix realization of spectral bounds
of the spectral radius of a nonnegative matrix}, Ph.D. Thesis, NCTU, 2018.

\bibitem{spec_rad}
A. E. Brouwer, W. H. Haemers, {\it Spectra of graphs}, Springer, 2012

\bibitem{prn_fros2}
R. A. Horn , C. R. Johnson, {\it Matrix analysis}, Cambrigde University Press, 1985.

\bibitem{sage}
[Sage] SageMath, the Sage Mathematics Software System (Version 9.2),
       The Sage Developers, 2020, https://www.sagemath.org.  %sage tutorial

\end{thebibliography}




\end{document}
