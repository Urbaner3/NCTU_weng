\documentclass[12pt]{report}%{article}

\topmargin 0pt

\topmargin=-1.5cm
\oddsidemargin=0.7cm
\textheight=23.5cm
\textwidth=15cm
\setlength{\baselineskip}{24pt}
\renewcommand{\baselinestretch}{1.5} %行距




\usepackage{fontspec}   %加這個就可以設定字體
\usepackage{xeCJK}       %讓中英文字體分開設置
%\usepackage{times}
\usepackage{amssymb}
\usepackage{amsmath}
\usepackage{amsthm}
\usepackage{ulem}
\usepackage{amsfonts}
\usepackage{mathrsfs}
\usepackage{tabularx,array}
\usepackage{pgf,tikz}
\usepackage{blkarray} %line 80-92 Q formula matrix index
%\usepackage{colortbl}

\usetikzlibrary{arrows}
\setCJKmainfont{標楷體} %設定中文為系統上的字型,而英文不去更動,使用原TeX字型
\XeTeXlinebreaklocale "zh"             %這兩行一定要加,中文才能自動換行
\XeTeXlinebreakskip = 0pt plus1pt     %這兩行一定要加,中文才能自動換行
\title{Combinatorial Identities from Lagrange's Interpolation Polynomial}
\author{Student: Yen-Jung Huang  ~~~~~~~~~~~~~~~~~~~~~~~~~~Advisor: Chih-Wen Weng}
\date{} %不要日期

\def\UrlFont{\rm}


\theoremstyle{plain}
\newtheorem{thm}{Theorem}[chapter]
\newtheorem{cor}[thm]{Corollary}
\newtheorem{lem}[thm]{Lemma}
\newtheorem{prop}[thm]{Proposition}
\newtheorem{remark}[thm]{Remark}
\newtheorem{eg}[thm]{Example}
\newtheorem{conj}[thm]{Conjecture}


\theoremstyle{definition}
\newtheorem{defn}[thm]{Definition}
\newtheorem{ex}[thm]{Exercise}
\newtheorem{prob}[thm]{Problem}
\newtheorem{exam}[thm]{Example}
\newtheorem{nota}[thm]{Notation}
\newtheorem{rem}[thm]{Remark}
\newtheorem{ques}[thm]{Question}
\newtheorem{pof}[thm]{Proof.}

%\renewcommand {\refname} {Bibliography}

\begin{document}

\chapter{Our result}
We will prove the main Theorems introduced in the introduction. First, we need some natations.
 Throughout fix $k \in [n-1] := \{1,2,...,n-1\}$. Let $E_{kn}$ denote the $n\times n$ binary matrix with an unique $1$ appearing in the  position $(k,n)$. We will apply the Theorem 1.7 with $P=I$ and 



\begin{equation}
Q=I+E_{kn}=\begin{blockarray}{cccccc}
%a & b & c & d & e \\
 &  &  & &  \\
\begin{block}{(ccccc)c}
	\label{Q1}
  1 &   &   &   &   &  \\
    & 1 &   &   &   &  \\
    &   & \ddots &   & 1 & k \\
    &   &   & 1 &   &  \\
  0 &   &   &   & 1 &  \\
\end{block}
\end{blockarray}
\end{equation}


\begin{defn}\label{v_rooted}
 ~A column vector $v'=(v'_1,v'_2,\ldots,v'_n)^T$ is called {\it $k$-rooted}  if $v'_{j} \geq 0$ for $1 \leq  j \leq n$ and $v'_k\geq v'_n.$
\end{defn}
\bigskip

The following Lemma is immediate from the above definition.
\bigskip

\begin{lem}
If ~$u=(u_1, u_2, \ldots, u_n)^T$ and $v'=(v'_1, v'_2, \ldots, v'_n):=Qu=(u_1,\ldots, u_{k-1},u_k+u_n, u_{k+1}, \ldots,  u_n)^T$, then
\begin{enumerate}
\item[(i)] $v'$ is $k$-rooted  if and only if  $u$ is nonnegative;
\item[(ii)] $u_k>0$ if and only if $v'_k>v'_n$.
\end{enumerate}
\qed
\end{lem}

\begin{thm} \label{thm_main}
    Let $C=(c_{ij})$, $C'=(c'_{ij})$ be  $n\times n$ matrices.
Assume that
\begin{enumerate}
\item[(i)]   $C[[n]|[n-1]]\leq C'[[n]|[n-1]]$ and $c_{ik}+c_{in}\leq c'_{ik}+c'_{in}$ for all $1\leq i\leq n$;
\item[(ii)] there exists a $k$-rooted vector $v'=(v'_1, v'_2, \ldots, v'_n)^T$ and a scalar $\lambda'\in \mathbb{R}$
such that $\lambda'$ is an eigenvalue of $C'$ with associated eigenvector $v'$;
\item[(iii)] there exists a nonnegative vector $v^T=(v_1, v_2, \ldots, v_n)$ and a scalar $\lambda\in \mathbb{R}$ such that $\lambda$ is an eigenvalue of $C$ with associated left eigenvector $v^T$;
\item[(iv)] $v^Tv'>0.$
\end{enumerate}
 Then $\lambda\leq \lambda'$.
Moreover, $\lambda=\lambda'$
if and only if
\begin{enumerate}
\item[(a)]    $c_{ik}+c_{in}=c'_{ik}+c'_{in} \qquad$  for $1\leq i\leq n$ with $v_i\not=0$ and $v'_n\not=0;$
\item[(b)]  $c'_{ij}=c_{ij}\qquad $for $1\leq i\leq n,~1\leq j\leq n-1, j \neq k $with $v_i\ne 0 $;
\item[(c)]   $c'_{ik}=c_{ik} \qquad $  for $1\leq i \leq n$ and $ v'_{k}>v'_n$ 
\end{enumerate} \qed


\end{thm}

\begin{proof}
The proof is based on Theorem 1.7 with $P = I$ and $Q = I + E_{kn} \text{ in } (\ref{Q1} )  $. The assumption (i) $PCQ\leq PC'Q$ of Theorem 1.7 holds by the condition (i) of this theorem. Let $u = Q^{-1}v'$. Then u is nonnegative and $C'Qu = \lambda' Qu$ by the condition (ii) and Lemma 2.2(i). Hence the assumption (ii) of Theorem 1.7 holds. The assumptions (iii) and (iv) of Theorem 1.7 clearly hold by conditions (iii),(iv) of this theorem since $P = I$ and $v'= Qu$  Hence $\lambda \leq \lambda' $ by the conclusion of Theorem 1.7. Moreover $\lambda = \lambda'$ if and only if \ref{pre0} holds, and this is equivalent to conditions (a),(b),(c) of this theorem. 
\end{proof}

We are interested in the matrices $C'$ that have $k$-rooted eigenvectors.
Motivated by the condition (i) of theorem 2.3, we provide the following two definitions. 

\begin{defn}
For an $n \times n$ matrix $C'=(c'_{ij})$, the $(k, n)$-{\it sum} vector of $C'$ is the vector of the sum of the $k$-th and  $n$-th columns of $C'$.
\end{defn}

Note that the last column of $C'Q$ is the $(k, n)$-{\it sum} vector of $C'$

\begin{defn}\label{m_rooted}
A  matrix $C'=(c'_{ij})$ is called {\it $k$-rooted}  if its  columns and its $(k, n)$-{\it sum} vector are all $k$-rooted except the last column of $C'$.
\end{defn}

The following theorem shows that a $k$-rooted matrix has a $k$-rooted eigenvector. 

\begin{thm} \label{lma_m_rooted}
Let $C'=(c'_{ij})$ be an $n\times n$ nonnegative matrix. Then the following (i)-(iii) hold.
    \begin{enumerate}
        \item[(i)]$C'$ is a $k$-rooted matrix, if and only if, $Q^{-1}C'Q$ is nonnegative.
        \item[(ii)]Assume that $C'$ is $k$-rooted and let $u$ be a nonnegative eigenvector of $Q^{-1}C'Q$ for $\rho(C')$. Then  $C'$ has a $k$-rooted eigenvector $v'=Qu$ for $\rho(C')$. 
        \item[(iii)] $\rho(C')$ = $\rho(Q^{-1}C'Q)$
    \end{enumerate}
\end{thm}

\begin{proof}
(i) is immediate from Definition~\ref{m_rooted} and the observation that
$$Q^{-1}=I-E_{kn}=\begin{pmatrix}
1 &  & & &  & 0 \\
 & 1 &  &      &  &  \\
 &  & \ddots & &  & -1 \\
 &  &        & &  &  \\
  &  & & & 1 &  \\
0 &  & & &  & 1 \\
\end{pmatrix},$$
and $Q^{-1}C'Q$ is
$$\begin{pmatrix}
c'_{11}     & c'_{12} & \cdots     & c'_{1\ n-1} & c'_{1k}+c'_{1n} \\
\vdots \\
c'_{k-11}     & c'_{k-1 2}           & \cdots     & c'_{k-1 n-1} & c'_{k-1k}+c'_{k-1n} \\
c'_{k1}-c'_{n1} & c'_{k2}-c'_{n2} &\cdots      &c'_{kn-1}-c'_{nk-1}& c'_{kk}+c'_{kn}-c'_{nk}-x'_{nn}\\
c'_{k+11}     & c'_{k+12}           & \cdots     & c'_{k+1\ n-1} & c'_{k+1k}+c'_{k+1n} \\
\vdots              & \vdots & \ddots              & \vdots & \vdots \\
c'_{n1}             & c'_{n2} & \cdots             & c'_{n\ n-1} & c'_{nk}+c'_{nn} \\
\end{pmatrix}.
$$



(ii)
By Lemma~\ref{v_rooted} $v'=Qu$ is $k$-rooted.
Since $Q^{-1}C'Qu=\rho(C')u$ by the assumption, we have
$Q^{-1} C' Q u  = Q^{-1} \rho(C') Qu  =\rho(C')u$
$C'Qu=\rho(C')Qu$.



(iii)
Since $C'$ and $Q^{-1}C'Q$ have the same set of eigenvalues, clearly $\rho(C')$ = $\rho(Q^{-1}C'Q)$.

\end{proof}


\begin{lem}\label{l_diag}
If a square matrix $C'$ has a rooted eigenvector for $\lambda'$, then $C'+dI$ also has
the same rooted eigenvector for $\lambda'+d,$ where $d$ is a constant and $I$ is the identity matrix with the same size of $C'$.
\end{lem}

\begin{thm}
Let $C$ be an $n\times n$ nonnegative Tmatrix. For $1\leq i \leq n$ and $1\leq j\leq n-1$, choose $c'_{ij}$
such that $c'_{ij}\geq c_{ij}$ and $c'_{kj}\geq c'_{nj}>0$, and choose $r'_i$ such that $r'_i\geq c_{ik}+c_{in}$, and
$r'_k \geq r'_n$. Moreover choose $c'_{in}:=r'_i-c'_{ik}$. Then $\rho(C)\leq \rho(C')$, when $C'=(c'_{ij})$.
\end{thm}



\begin{proof}

The assumptions are necessary that $PCQ \leq PC'Q$, and C' is $k$-rooted,based on (\ref{lma_m_rooted}) ; by \ref{l_diag}, For certain d, if $C'+dI$ is $k$-rooted, then it has a $k$-rooted eigenvector with its spectral radius $\lambda + d$. C' would share the same eigenvector with $C'+dI$ and has eigenvalue $\lambda$. So $C'+dI$ and $C+dI$ meet the conditions of \ref{thm_main}, and we can show that $\rho(C' + dI) \geq \rho(C +dI)$ and then $\rho(C') \geq \rho(C)$  \qed



\noindent{\bf Example}
For the following $4\times 4$ matrix
$$C=\begin{pmatrix}
0 & 0 & 1 & 1\\
1 & 0 & 0 & 1\\
1 & 1 & 0 & 1\\
1 & 1 & 1 & 0
\end{pmatrix},$$
we choose
$$C'=\begin{pmatrix}
0 & 0 & 1 & 1\\
1 & 0 & 1 &  0\\
1 & 1 & 0 & 1\\
1 & 1 & 1 & 0
\end{pmatrix}.$$
Check the conditions $C[[n]|[n-1]]  \leq C'[[n]|[n-1]] $ \ref{thm_main},   
Then
$\rho(C)\leq \rho(C')$ by Theorem 4.8.




{\bf Counterexample}
For the following two $4\times 4$ matrices
$$C=\begin{pmatrix}
0 & 0 & 1 & 1\\
1 & 0 & 0 & 1\\
1 & 1 & 0 & 0\\
1 & 1 & 1 & 0
\end{pmatrix},\quad C'=\begin{pmatrix}
0 & 0 & 1 & 1\\
1 & 0 & 1 &  0\\
1 & 1 & 0 & 0\\
1 & 1 & 1 & 0
\end{pmatrix},$$
specify $n$=4, $k$=3 in $Q = I +E_{kn} = I + E_{34}$  
$$CQ=\begin{pmatrix}
0 & 0 & 1 & 2\\
1 & 0 & 0 & 1\\
1 & 1 & 0 & 0\\
1 & 1 & 1 & 1
\end{pmatrix},\quad C'Q=\begin{pmatrix}
0 & 0 & 1 & 2\\
1 & 0 & 1 & 1\\
1 & 1 & 0 & 0\\
1 & 1 & 1 & 1
\end{pmatrix},$$

we have $CQ\leq C'Q$, but
$\rho(C)=2.234\not\leq 2.148= \rho(C')$.
This is because $c'_{33}+c'_{34}\not\geq c'_{43}+c'_{44}$.

\end{proof}

\end{document}